\outline{1} {Chapter 3 Theorems in First-Order Theories}
\centerline{\xmplbxi CHAPTER 3}
\medskip
\centerline{\xmplbx THEOREMS IN FIRST-ORDER THEORIES}
\bigskip

% ==============================================================================
\outline{2} {Definitions}
\beginsection DEFINITIONS

\item{$\bullet$} $\A$ is {\it elementary} if it is either atomic or an instantiation.

\item{$\bullet$} A {\it truth valuation} for $T$ is a mapping from the set of elementary 
formulas in $T$ to the set of truth values.

\item{$\bullet$} $\B$ is a {\it tautological consequence} of $\A_1, \dots , \A_n$ 
if $V(\B) = \top$ for every truth valuation $V$ such that 
$V(\A_1) = \cdots = V(\A_n) = \top$.

\item{$\bullet$} $\A$ is a {\it tautology} if it is a tautological consequence of 
the empty sequence of formulas, i.e. if $V(\A) = \top$ for every truth valuation $V$.

\item{$\bullet$} $\A'$ is an {\it instance} of $\A$ if $\A'$ is of the form $\A_{\x_1, \dots, \x_n}[\a_1, \dots, \a_2]$.

\item{$\bullet$} Let $\A$ be a formula and $\x_1, \dots, \x_n$ its free variables in alphabetical order. The {\it closure}
of $\A$ is the formula $\forall \x_1 \dots \forall \x_n \A$.

\item{$\bullet$} $\A'$ is a {\it variant} of $\A$ if $\A'$ can be obtained from $\A$ by a sequence of replacements of the
following type: replace a part $\exists \x\B$ by $\exists \y\B_\x[\y]$, where $\y$ is a variable not free in $\B$.

\item{$\bullet$} $\A$ is {\it open} if it contains no quantifiers.

\item{$\bullet$} $\A$ is in {\it prenex form} if it has the form $Q\x_1 \dots Q\x_n\B$ 
where each $Q\x_i$ is either $\exists\x_i$ or $\forall\x_i$; $\x_1, \dots, \x_n$ are
distinct; and $\B$ is open.

% ==============================================================================
\outline{2} {Results}
\beginsection RESULTS

\noindent {\bf \S3.1}

\anchor{3.1.tautology}
\proclaim Tautology Theorem. If $\B$ is a tautological consequence of $\A_1, \dots, \A_n$,
and $\vdash\A_1, \dots, \vdash \A_n$, then $\vdash \B$.

\proclaim Corollary. Every tautology is a theorem.

\proclaim Lemma 1. If $\vdash \A \lor \B$, then $\vdash \B \lor \A$.

\proclaim Detachment Rule. If $\vdash \A$ and $\vdash \A \to \B$, then $\vdash \B$.

\proclaim Corollary. If $\vdash \A_1, \dots, \vdash \A_n$, and $\vdash \A_1 \to \dots \to \A_n \to \B$, then $\vdash \B$.

\proclaim Lemma 2. If $n \ge 2$, and $\A_1 \lor \cdots \lor \A_n$ is a tautology, then $\vdash \A_1 \lor \cdots \lor \A_n$.

\noindent {\bf \S3.2}

\proclaim $\forall$-Introduction Rule. If $\vdash \A \to \B$ and $\x$ is not free in $\A$, then $\vdash \A \to \forall\x\B$.

\proclaim Generalization Rule. If $\vdash \A$, then $\vdash \forall\x\A$.

\proclaim Substitution Rule. If $\vdash \A$ and $\A'$ is an instance of $\A$, then $\vdash \A'$.

\proclaim Substitution Theorem. 
$$\eqalign{&\vdash \A_{\x_1, \dots, \x_n}[\a_1, \dots, \a_n] \to \exists \x_1 \dots \exists \x_n \A\cr
&\vdash \forall \x_1 \dots \forall \x_n \A \to \A_{\x_1, \dots, \x_n}[\a_1, \dots, \a_n]}$$

\proclaim Distribution Rule. If $\vdash \A \to \B$, then $\vdash \exists\x\A \to \exists\x\B$ and $\vdash \forall\x\A \to \forall\x\B$.

\proclaim Closure Theorem. If $\A'$ is the closure of $\A$, then $\vdash \A'$ iff $\vdash \A$.

\proclaim Corollary. If $\A'$ is the closure of $\A$, then $\A'$ is valid in a structure $\cal A$ iff $\A$ is valid in $\cal A$.

\noindent {\bf \S3.3}

\proclaim Deduction Theorem. Let $\A$ be a closed formula in $T$. For every formula $\B$ of $T$, $\vdash_T \A \to \B$ iff $\B$ is a theorem of $T[\A]$.

\proclaim Corollary. Let $\A_1, \dots, \A_n$ be closed formulas in $T$. For every formula $\B$ in $T$, $\vdash_{T} \A_1 \to \cdots \to \A_n \to \B$ 
iff $\B$ is a theorem of $T[\A_1, \dots, \A_n]$.

\proclaim Theorem on Constants. Let $T'$ be obtained from $T$ by adding new constants (but no new nonlogical axioms).
For every formula $\A$ of $T$ and every sequence $\e_1, \dots, \e_n$ of distinct new constants, 
$\vdash_T \A$ iff $\vdash_{T'} \A[\e_1, \dots, \e_n]$.

\noindent {\bf \S3.4}

\proclaim Equivalence Theorem. Let $\A'$ be obtained from $\A$ by replacing some occurrences of $\B_1, \dots, \B_n$
by $\B_1', \dots, \B_n'$, respectively. If
$$
\vdash \B_1 \leftrightarrow \B_1', \dots, \vdash \B_n \leftrightarrow \B_n'
$$
then
$$
\vdash \A \leftrightarrow \A'.
$$

\proclaim Variant Theorem. If $\A'$ is a variant of $\A$, then $\vdash \A \leftrightarrow \A'$.

\proclaim Symmetry Theorem. $\vdash \a = \b \leftrightarrow \b = \a$.

\proclaim Equality Theorem. Let $\b'$ be obtained from $\b$ by replacing some occurrences of $\a_1, \dots, \a_n$
not immediately following $\exists$ or $\forall$ by $\a_1', \dots, \a_n'$ respectively, and let $\A'$ be
obtained from $\A$ by the same type of replacements. If $\vdash \a_1 = \a_1', \dots, \vdash \a_n = \a_n'$ then
$\vdash \b = \b'$ and $\vdash \A \leftrightarrow \A'$.

\proclaim Corollary 1. $\vdash \a_1 = \a_1' \to \cdots \to \a_n = \a_n' \to \b[\a_1, \dots, \a_n] = \b[\a_1', \dots, \a_n']$.

\proclaim Corollary 2. $\vdash \a_1 = \a_1' \to \cdots \to \a_n = \a_n' \to (\A[\a_1, \dots, \a_n] \leftrightarrow \A[\a_1', \dots, \a_n'])$.

\proclaim Corollary 3. If $\x$ does not occur in $\a$, then
$$
\vdash \A_\x[\a] \leftrightarrow \exists\x (\x=\a \land \A)
$$

\vfill
\break

% ==============================================================================
\outline{2} {Problems}
\beginsection PROBLEMS

% ------------------------------------------------------------------------------
\ans 1. Let's prove it by induction on theorems (as in \S3.1).
If $\A$ is a theorem provable without use of substitution axioms, 
identity axioms, equality axioms, nonlogical axioms or the $\exists$-introduction 
rule, then it is a tautological consequence of some theorems $\B_1, \dots, \B_n$.
If $n=0$, then $\A$ is a tautology, since it's a tautological consequence of the
empty sequence of formulas. Otherwise, by the induction hypothesis, if 
$\B_1, \dots, \B_n$ can be proven without the use of substitution axioms, identity 
axioms, equality axioms, nonlogical axioms or the $\exists$-introduction rule, 
they are also tautologies. This means that $V(\B_i) = \top$ for all $i$ and truth
valuations $V$, which implies that $V(\A) = \top$ for all truth valuations and thus
$\A$ is also a tautology.
\medskip

% ------------------------------------------------------------------------------
\ans 2. First note, that if a formula $\A$ is a tautology, then it's a tautological 
consequence of any set of formulas $\Gamma$ since $V(\A) = \top$ for every truth 
valuation $V$. We use induction on theorems (as in \S3.1):
\item{$\bullet$} If $\A$ is an identity axiom $\x = \x$, then $\A^*$ is $\e = \e$ 
which is a tautological consequence of $\e = \e$.
\item{$\bullet$} If $\A$ is an equality axiom 
$$\x_1 = \y_1 \to \cdots \to \x_n = \y_n \to \f\x_1 \dots \x_n = \f\y_1 \dots \y_n,$$
then $\A^*$ is 
$$\e = \e \to \cdots \to \e = \e \to \e = \e,$$
which is a tautology.
\item{$\bullet$} If $\A$ is an equality axiom 
$$\x_1 = \y_1 \to \cdots \to \x_n = \y_n \to \p\x_1 \dots \x_n \to \p\y_1 \dots \y_n,$$
then $\A^*$ is 
$$\e = \e \to \cdots \to \e = \e \to \p\e \dots \e \to \p\e \dots \e,$$
which is a tautology.
\item{$\bullet$} If $\A$ is a substitution axiom $\B_\x[\a] \to \exists\x\B$, then 
$\A^*$ is $(\B_\x[\a])^* \to \B^*$. First, let's show that $(\B_\x[\a])^*$ and $\B^*$ 
are the same, by induction on the length of $\B$:
\itemitem{$\circ$} If $\B$ is an atomic formula $\p\a_1 \dots \a_n$, then any occurrence 
of $\x$ in $\B$ is completely inside one of the $\a_i$. Hence, both $(\B_\x[\a])^*$ and 
$\B^*$ are $\p\e \dots \e$.
\itemitem{$\circ$} If $\B$ is $\neg \C$, then $(\B_\x[\a])^*$ is $\neg (\C_\x[\a])^*$,
$\B^*$ is $\neg \C^*$, and $(\C_\x[\a])^*$ and $\C^*$ are the same by the induction 
hypothesis.
\itemitem{$\circ$} If $\B$ is $\C \lor \D$, then $(\B_\x[\a])^*$ is $(\C_\x[\a])^* \lor (\D_\x[\a])^*$
and $\B^*$ is $\C^* \lor \D^*$. Finally, $(\C_\x[\a])^*$ and $\C^*$ are the same, and
$(\D_\x[\a])^*$ and $\D^*$ are the same, by the induction hypothesis.
\itemitem{$\circ$} Suppose $\B$ is $\exists\y\C$: if $\x$ and $\y$ are the same, then
obviously $\B_\x[\a]$ and $\B$ are the same since $\B$ has no free occurrences of $\x$.
Otherwise, $(\B_\x[\a])^*$ is $(\C_\x[\a])^*$, $\B^*$ is $\C^*$, and $(\C_\x[\a])^*$ and
$\C^*$ are the same by the induction hypothesis.

Hence, $\A^*$ is $\B^* \to \B^*$ which is a tautology.
\item{$\bullet$} If $\A$ was obtained from $\B \to \C$ by the $\exists$-introduction rule,
then $\A$ is $\exists\x\B \to \C$, $\A^*$ is $\B^* \to \C^*$, and $\B^* \to \C^*$ is 
a tautological consequence of $\e = \e$ by the induction hypothesis.
\item{$\bullet$} If $\A$ is a tautological consequence of formulas $\B_1, \dots, \B_n$, 
then the $\B_1^*, \dots, \B_n^*$ are tautological consequences of $\e=\e$ by the
induction hypothesis. Hence, $\A^*$ is a tautological consequence of 
$\B_1^*, \dots, \B_n^*$ (see the {\it Remark} in \S3.1, page 30), so $\A^*$ is also
a tautological consequence of $\e = \e$.

Now, let's assume that there is a formula $\A$ such that $\vdash_{T} \A$ and $\vdash_{T} \neg \A$.
Using the result just proved, the first condition implies that $\A^*$ is a tautological
consequence of $\e=\e$ and the second condition implies that $\neg \A^*$ is also a 
tautological consequence of $\e=\e$. Hence, for any truth valuation $V$ such that 
$V(\e=\e) = \top$, we must have $V(\A^*) = \top$ and $V(\neg \A^*) = \neg V(\A^*) = \top$.
This is a contradiction. Thus, there is no such formula.

This result is interesting, since it basically says that we can not prove two theorems 
that contradict each other in $T$.
\medskip

% ------------------------------------------------------------------------------
\ans 3. 

\ansitem (a)
Let $\x_1, \dots, \x_n$ the free variables of $\forall\x(\A \to \B)$ and $\exists\x\A \to \exists\x\B$; 
let $T'$ be a theory obtained from $T$ by adding $n$ new constants $\e_1, \dots, \e_n$; 
let $\C$ be $\forall\x(\A \to \B)$ and let $\D$ be $\exists\x\A \to \exists\x\B$.
Note that
$$
\vdash_T \C \to \D \quad\hbox{iff}\quad \vdash_{T'} \C[\e_1, \dots, \e_n] \to \D[\e_1, \dots, \e_n]
$$
by the Theorem on Constants, and
$$
\vdash_{T'} \C[\e_1, \dots, \e_n] \to \D[\e_1, \dots, \e_n] \quad\hbox{iff}\quad \vdash_{T'[\C[\e_1, \dots, \e_n]]} \D[\e_1, \dots, \e_n] 
$$
by the Deduction Theorem. Hence, in $T'[\C[\e_1, \dots, \e_n]]$, we have
\item{} $\vdash \C[\e_1, \dots, \e_n]$ \hfill [the added nonlogical axiom]
\item{} $\vdash \forall\x(\A[\e_1, \dots, \e_n] \to \B[\e_1, \dots, \e_n])$ \hfill [by the definition of $\C$]
\item{} $\vdash \forall\x(\A[\e_1, \dots, \e_n] \to \B[\e_1, \dots, \e_n]) \to (\A[\e_1, \dots, \e_n] \to \B[\e_1, \dots, \e_n])$ \hfill [Substitution Theorem]
\item{} $\vdash \A[\e_1, \dots, \e_n] \to \B[\e_1, \dots, \e_n]$ \hfill [Detachment Rule]
\item{} $\vdash \exists\x\A[\e_1, \dots, \e_n] \to \exists\x\B[\e_1, \dots, \e_n]$ \hfill [Distribution Rule]
\item{} $\vdash \D[\e_1, \dots, \e_n]$ \hfill [by the definition of $\D$]
\smallskip 

\ansitem (b) As in (a), but using the universal-quantifier form of the Distribution Rule.
\medskip

% ------------------------------------------------------------------------------
\ans 4.
\ansitem (a) For the forward implication we have
\itemitem{(1)} $\vdash \A \to \exists\x\A$ \hfill [Substitution Theorem]
\itemitem{(2)} $\vdash \B \to \exists\x\B$ \hfill [Substitution Theorem]
\itemitem{(3)} $\vdash \A \lor \B \to \exists\x\A \lor \exists\x\B$ \hfill [from (1) and (2) by the Tautology Theorem]
\itemitem{(4)} $\vdash \exists\x(\A \lor \B) \to \exists\x\A \lor \exists\x\B$ \hfill [from (3) by the $\exists$-Introduction Rule]

and for the reverse implication we have
\itemitem{(1)} $\vdash \A \to \A \lor \B$ \hfill [Tautology Theorem]
\itemitem{(2)} $\vdash \B \to \A \lor \B$ \hfill [Tautology Theorem]
\itemitem{(3)} $\vdash \exists\x\A \to \exists\x(\A \lor \B)$ \hfill [from (1) by the Distribution Rule]
\itemitem{(4)} $\vdash \exists\x\B \to \exists\x(\A \lor \B)$ \hfill [from (2) by the Distribution Rule]
\itemitem{(5)} $\vdash \exists\x\A \lor \exists\x\B \to \exists\x(\A \lor \B)$ \hfill [from (3) and (4) by the Tautology Theorem]
\smallskip 

\ansitem (b) For the forward implication we have
\itemitem{(1)} $\vdash \A \land \B \to \A$ \hfill [Tautology Theorem]
\itemitem{(2)} $\vdash \A \land \B \to \B$ \hfill [Tautology Theorem]
\itemitem{(3)} $\vdash \forall\x(\A \land \B) \to \forall\x\A$ \hfill [from (1) by the Distribution Rule]
\itemitem{(4)} $\vdash \forall\x(\A \land \B) \to \forall\x\B$ \hfill [from (2) by the Distribution Rule]
\itemitem{(5)} $\vdash \forall\x(\A \land \B) \to \forall\x\A \land \forall\x\B$ \hfill [from (3) and (4) by the Tautology Theorem]

and for the reverse implication we have
\itemitem{(1)} $\vdash \forall\x\A \to \A$ \hfill [Substitution Theorem]
\itemitem{(2)} $\vdash \forall\x\B \to \B$ \hfill [Substitution Theorem]
\itemitem{(3)} $\vdash \forall\x\A \land \forall\x\B \to \A \land \B$ \hfill [from (1) and (2) by the Tautology Theorem]
\itemitem{(4)} $\vdash \forall\x\A \land \forall\x\B \to \forall\x(\A \land \B)$ \hfill [from (3) by the $\forall$-Introduction Rule]
\smallskip 

\ansitem (c)
\itemitem{(1)} $\vdash \A \to \exists\x\A$ \hfill [Substitution Theorem]
\itemitem{(2)} $\vdash \B \to \exists\x\B$ \hfill [Substitution Theorem]
\itemitem{(3)} $\vdash \A \land \B \to \exists\x\A \land \exists\x\B$ \hfill [from (1) and (2) by the Tautology Theorem]
\itemitem{(4)} $\vdash \exists\x(\A \land \B) \to \exists\x\A \land \exists\x\B$ \hfill [from (3) by the $\exists$-Introduction Rule]
\smallskip 

\ansitem (d)
\itemitem{(1)} $\vdash \forall\x\A \to \A$ \hfill [Substitution Theorem]
\itemitem{(2)} $\vdash \forall\x\B \to \B$ \hfill [Substitution Theorem]
\itemitem{(3)} $\vdash \forall\x\A \lor \forall\x\B \to \A \lor \B$ \hfill [from (1) and (2) by the Tautology Theorem]
\itemitem{(4)} $\vdash \forall\x\A \lor \forall\x\B \to \forall\x(\A \lor \B)$ \hfill [from (3) by the $\forall$-Introduction Rule]
\smallskip 

\ansitem (e) Consider the formula
$$
\forall x (x=0 \lor 0 < x) \to \forall x (x=0) \lor \forall x (0 < x).
$$
The left side says {\it ``every number is either zero or greater than zero"} while 
the right side says {\it ``every number is zero or every number is greater than zero"}.

Then, consider the formula
$$
\exists x (x=0) \land \exists x (0 < x) \to \exists x (x=0 \land 0 < x).
$$
The left side says {\it ``there is a number which is equal to zero and there is 
a number that is greater than zero"} while the right side says {\it ``there is a number
which is both equal to zero and greater than zero"}.


\medskip

% ------------------------------------------------------------------------------
\ans 5. The existential form
\itemitem{(1)} $\vdash \A \to \exists\x\A$ \hfill [Substitution Theorem or Substitution Axiom]
\itemitem{(2)} $\vdash \A \to \A$ \hfill [Propositional Axiom and definition of $\to$]
\itemitem{(3)} $\vdash \exists\x\A \to \A$ \hfill [from (2) by the $\exists$-Introduction Rule]
\itemitem{(4)} $\vdash \exists\x\A \leftrightarrow \A$ \hfill [from (1) and (3) and the definition of $\leftrightarrow$]

\noindent and the universal form
\itemitem{(1)} $\vdash \forall\x\A \to \A$ \hfill [Substitution Theorem]
\itemitem{(2)} $\vdash \A \to \A$ \hfill [Propositional Axiom and definition of $\to$]
\itemitem{(3)} $\vdash \A \to \forall\x\A$ \hfill [from (2) by the $\forall$-Introduction Rule]
\itemitem{(4)} $\vdash \forall\x\A \leftrightarrow \A$ \hfill [from (1) and (3) and the definition of $\leftrightarrow$]

\medskip

% ------------------------------------------------------------------------------
\ans 6. 

\ansitem (a) 
\itemitem{} $\vdash \A \to \exists\x\exists\y\A$ \hfill [Substitution Theorem]
\itemitem{} $\vdash \exists\x\A \to \exists\x\exists\y\A$ \hfill [$\exists$-Introduction Rule]
\itemitem{} $\vdash \exists\y\exists\x\A \to \exists\x\exists\y\A$ \hfill [$\exists$-Introduction Rule]

The reverse implication is obtained in a similar fashion. Note that it is also possible to obtain 
$\exists\y\exists\x\A$ as a variant of $\exists\x\exists\y\A$: first obtain $\exists\x\exists\x'\A'$
where $\A'=\A_y[\x']$ and $\x'$ is a new variable not appearing in $\A$; then obtain $\exists\y\exists\x'\A'_\x[\y]$ 
and finally $\exists\y\exists\x\A$ by substituting $\x'$ to $\x$. This would imply the result by the
Variant Theorem.
\smallskip

\ansitem (b)
\itemitem{} $\vdash \forall\x\forall\y\A \to \A$ \hfill [Substitution Theorem]
\itemitem{} $\vdash \forall\x\forall\y\A \to \forall\x\A$ \hfill [$\forall$-Introduction Rule]
\itemitem{} $\vdash \forall\x\forall\y\A \to \forall\y\forall\x\A$ \hfill [$\forall$-Introduction Rule]

The reverse implication is obtained in a similar fashion.
\smallskip

\ansitem (c)
\itemitem{} $\vdash \A \to \exists\x\A$ \hfill [Substitution Theorem]
\itemitem{} $\vdash \forall\y\A \to \forall\y\exists\x\A$ \hfill [Distribution Rule]
\itemitem{} $\vdash \exists\x\forall\y\A \to \forall\y\exists\x\A$ \hfill [$\exists$-Introduction Rule]

Note that it might seem that using the dual results in the above proof, the opposite implication could be
obtained (i.e. $\vdash \forall\y\exists\x\A \to \exists\x\forall\y\A$). However this is not the case, as
they result in an alternative proof of the same result as above:

\itemitem{} $\vdash \forall\y\A \to \A$ \hfill [Substitution Theorem]
\itemitem{} $\vdash \exists\x\forall\y\A \to \exists\x\A$ \hfill [Distribution Rule]
\itemitem{} $\vdash \exists\x\forall\y\A \to \forall\y\exists\x\A$ \hfill [$\forall$-Introduction Rule]
\smallskip

\ansitem (d) Consider the formula
$$\forall x \exists y (Sx = y) \to \exists y \forall x (Sx = y).$$
The left side can be interpreted as {\it ``every number has a successor"}, while the
right side can be interpreted as {\it ``there is a number that is the successor of every number"}.

\medskip

% ------------------------------------------------------------------------------
\ans 7. 

\ansitem (a) Let $\y_1, \dots, \y_m$ include the variables free in $\forall\x_1 \dots \forall\x_n (\B \leftrightarrow \B')$ 
and let $T'$ be obtained from $T$ by adding new constants $\e_1, \dots, \e_m$. Then 
$$
\vdash_T \forall\x_1 \dots \forall\x_n (\B \leftrightarrow \B') \to (\A \leftrightarrow \A')
$$
if and only if 
$$
\vdash_{T'} \forall\x_1 \dots \forall\x_n (\B[\e_1, \dots, \e_m] \leftrightarrow \B'[\e_1, \dots, \e_m]) \to (\A[\e_1, \dots, \e_m] \leftrightarrow \A'[\e_1, \dots, \e_m])
$$
by the Theorem on Constants. And furthermore
$$
\vdash_{T'} \forall\x_1 \dots \forall\x_n (\B[\e_1, \dots, \e_m] \leftrightarrow \B'[\e_1, \dots, \e_m]) \to (\A[\e_1, \dots, \e_m] \leftrightarrow \A'[\e_1, \dots, \e_m])
$$
if and only if
$$
\vdash_{T'[\forall\x_1 \dots \forall\x_n (\B[\e_1, \dots, \e_m] \leftrightarrow \B'[\e_1, \dots, \e_m])]} \A[\e_1, \dots, \e_m] \leftrightarrow \A'[\e_1, \dots, \e_m]
$$
by the Deduction Theorem. We can obtain the latter result as follows

\itemitem{} $\vdash \forall\x_1 \dots \forall\x_n (\B[\e_1, \dots, \e_m] \leftrightarrow \B'[\e_1, \dots, \e_m])$ \hfill [the added axiom]
\itemitem{} $\vdash \B[\e_1, \dots, \e_m] \leftrightarrow \B'[\e_1, \dots, \e_m]$ \hfill [Closure Theorem]
\itemitem{} $\vdash \A[\e_1, \dots, \e_m] \leftrightarrow \A'[\e_1, \dots, \e_m]$ \hfill [Equivalence Theorem]

Note that the universal quantifiers on $\x_1, \dots, \x_n$ are really necessary. For if a variable
appears free in $\B$ or $\B'$ and bound in $\A$ or $\A'$, the original occurrences of $\B$ in $\A$
are not necessarily the occurrences of $\B[\e_1, \dots, \e_m]$ in $\A[\e_1, \dots, \e_m]$ and $\A'$ 
would not be obtained from $\A$ as described. This would mean the Equivalence Theorem is not applicable.

\smallskip

\ansitem (b) Consider the formula
$$
(y<Sx \leftrightarrow x=2 \cdot y) \to (\exists y(y<Sx) \leftrightarrow \exists y (x=2 \cdot y))
$$
and an instance with $x=3, y=4$. Then
$$\eqalign{\SA(4 < 4) &= \bot\cr
\SA(3=2 \cdot 4) &= \bot\cr
\SA(\exists y (y < 4)) &= \top\cr
\SA(\exists y (3=2 \cdot y)) &= \bot\cr
}$$
and hence the formula is not valid in \N.

\medskip

% ------------------------------------------------------------------------------
\ans 8.

\ansitem (a) The proof parallels 7(a). Let $\y_1, \dots, \y_m$ include the variables 
free in $\forall\x_1 \dots \forall\x_n (\a = \a')$ and let $T'$ be obtained from $T$ 
by adding new constants $\e_1, \dots, \e_m$. Then 
$$
\vdash_T \forall\x_1 \dots \forall\x_n (\a = \a') \to (\A \leftrightarrow \A')
$$
if and only if 
$$
\vdash_{T'} \forall\x_1 \dots \forall\x_n (\a[\e_1, \dots, \e_m] = \a'[\e_1, \dots, \e_m]) \to (\A[\e_1, \dots, \e_m] \leftrightarrow \A'[\e_1, \dots, \e_m])
$$
by the Theorem on Constants. And furthermore
$$
\vdash_{T'} \forall\x_1 \dots \forall\x_n (\a[\e_1, \dots, \e_m] = \a'[\e_1, \dots, \e_m]) \to (\A[\e_1, \dots, \e_m] \leftrightarrow \A'[\e_1, \dots, \e_m])
$$
if and only if
$$
\vdash_{T'[\forall\x_1 \dots \forall\x_n (\a[\e_1, \dots, \e_m] = \a'[\e_1, \dots, \e_m])]} \A[\e_1, \dots, \e_m] \leftrightarrow \A'[\e_1, \dots, \e_m]
$$
by the Deduction Theorem. We can obtain the latter result as follows

\itemitem{} $\vdash \forall\x_1 \dots \forall\x_n (\a[\e_1, \dots, \e_m] = \a'[\e_1, \dots, \e_m])$ \hfill [the added axiom]
\itemitem{} $\vdash \a[\e_1, \dots, \e_m] = \a'[\e_1, \dots, \e_m]$ \hfill [Closure Theorem]
\itemitem{} $\vdash \A[\e_1, \dots, \e_m] \leftrightarrow \A'[\e_1, \dots, \e_m]$ \hfill [Equality Theorem]

Note that the universal quantifiers on $\x_1, \dots, \x_n$ are really necessary. For if a variable
appears in $\a$ or $\a'$ and is bound in $\A$ or $\A'$, the original occurrences of $\a$ in $\A$
are not necessarily the occurrences of $\a[\e_1, \dots, \e_m]$ in $\A[\e_1, \dots, \e_m]$ and $\A'$ 
would not be obtained from $\A$ as described. This would mean the Equality Theorem is not applicable.

\smallskip

\ansitem (b) Consider the formula
$$
(Sx = 2 \cdot x) \to (\exists x (Sx=y)) \leftrightarrow \exists x (2 \cdot x = y)
$$
and an instance with $x=1, y=3$. Then
$$\eqalign{\SA(2 = 2 \cdot 1) &= \top\cr
\SA(\exists x (Sx = 3)) &= \top\cr
\SA(\exists x (2 \cdot x = 3)) &= \bot\cr
}$$
and hence the formula is not valid in \N.

\medskip

% ------------------------------------------------------------------------------
\ans 9. First, notice that any formula $\x = \y \to \A \to \A_\x[\y]$ for $\A$
atomic, is a theorem of $T$: this follows from 8(a) and the Tautology Theorem
by noticing that $\A$ has no bound variables and thus no universal quantifiers
are needed. So $T$ has the same or more theorems than $T'$.

To prove that this relation is tight (i.e. that $T$ and $T'$ have the same theorems), 
let's show that any formula that is an equality axiom in $T$, is a theorem of $T'$.
Note that all the results from this chapter (except the Symmetry Theorem, the Equality
Theorem and its corollaries) can be proven without using the equality axioms and 
hence are also applicable to $T'$.

First, let's handle the equality axioms of the form
$$
\x_1=\y_1 \to \cdots \to \x_n=\y_n \to \p\x_1\dots\x_n \to \p\y_1\dots\y_n.
$$
Using the new axiom, we can obtain
$$\eqalign{\vdash \x_1=\y_1 &\to \p\x_1\y_2\dots\y_n \to \p\y_1\dots\y_n\cr
\vdash \x_2=\y_2 &\to \p\x_1\x_2\y_3\dots\y_n \to \p\x_1\y_2\dots\y_n\cr
&\cdots\cr
\vdash \x_n=\y_n &\to \p\x_1\dots\x_n \to \p\x_1\dots\x_{n-1}\y_n\cr
}$$
and then, by the Tautology Theorem
$$\eqalign{\vdash \p\x_1\y_2\dots\y_n \to \x_1=\y_1 &\to \p\y_1\dots\y_n\cr
\vdash \p\x_1\x_2\y_3\dots\y_n \to \x_2=\y_2 &\to \p\x_1\y_2\dots\y_n\cr
&\cdots\cr
\vdash \p\x_1\dots\x_n \to \x_n=\y_n &\to \p\x_1\dots\x_{n-1}\y_n.\cr
}$$
Also by the Tautology Theorem, if $\vdash \A \to \B \to \C$ and $\vdash \C \to \D$,
then $\vdash \A \to \B \to \D$. Using this and starting from the first formula, we
find successively
$$\eqalign{\vdash \p\x_1\x_2\y_3\dots\y_n &\to \x_2=\y_2 \to \x_1=\y_1 \to \p\y_1\dots\y_n\cr
\vdash \p\x_1\x_2\x_3\y_4\dots\y_n &\to \x_3=\y_3 \to \x_2=\y_2 \to \x_1=\y_1 \to \p\y_1\dots\y_n\cr
&\cdots\cr
\vdash \p\x_1\dots\x_n &\to \x_n=\y_n \to \cdots \to \x_1=\y_1 \to \p\y_1\dots\y_n\cr
}$$
and the equality axiom for predicates follows from the last formula and the Tautology Theorem.
Note that the Symmetry Theorem can now be proven in $T'$.

Now let's turn our attention to the equality axioms of the form
$$
\x_1=\y_1 \to \cdots \to \x_n=\y_n \to \f\x_1\dots\x_n = \f\y_1\dots\y_n.
$$
We need the following results
\itemitem{(i)} If $\vdash \A \to \A'$ and $\vdash \B \to \B'$, then $\vdash \A \to \B \to (\A' \land \B')$
\itemitem{(ii)} $\vdash (\a=\b \land \b=\c) \to \a=\c$

\smallskip
\noindent It is easy to verify that (i) follows from the Tautology Theorem. Let's prove (ii):
\itemitem{(1)} $\vdash \y=\x \to \y=\z \to \x=\z$ \hfill [by the new axiom]
\itemitem{(2)} $\vdash \b=\a \to \b=\c \to \a=\c$ \hfill [from (1) by the Substitution Rule]
\itemitem{(3)} $\vdash \a=\b \to \b=\c \to \a=\c$ \hfill [from (2) by the Symmetry Theorem and the Equivalence Theorem]
\itemitem{(4)} $\vdash \neg(\a=\b) \lor \neg(\b=\c) \lor (\a=\c)$ \hfill [from (3) and the definition of $\to$]
\itemitem{(5)} $\vdash (\neg(\a=\b) \lor \neg(\b=\c)) \lor (\a=\c)$ \hfill [from (4) by the Associative Rule]
\itemitem{(6)} $\vdash \neg\neg(\neg(\a=\b) \lor \neg(\b=\c)) \lor (\a=\c)$ \hfill [from (5) by the Tautology Theorem]
\itemitem{(7)} $\vdash (\a=\b \land \b=\c) \to (\a=\c)$ \hfill [from (6) and the definition of $\land$ and $\to$]

\smallskip
\noindent It essentially says that equality is transitive.

Now let's move on to the main part of the proof. First, we obtain the following 
result for any $i$ in $1 \dots n$
\itemitem{(1)} $\vdash \x_i=\y_i \to \f\z_1\dots\z_n = \f\x_1\dots\x_n \to \f\z_1\dots\z_n = \f\x_1\dots\x_{i-1}\y_i\x_{i+1}\dots\x_n$ \hfill [by the new axiom]
\itemitem{(2)} $\vdash \x_i=\y_i \to \f\x_1\dots\x_n = \f\x_1\dots\x_n \to \f\x_1\dots\x_n = \f\x_1\dots\x_{i-1}\y_i\x_{i+1}\dots\x_n$ \hfill [from (1) by the Substitution Rule]
\itemitem{(3)} $\vdash \f\x_1\dots\x_n = \f\x_1\dots\x_n \to \x_i=\y_i \to \f\x_1\dots\x_n = \f\x_1\dots\x_{i-1}\y_i\x_{i+1}\dots\x_n$ \hfill [from (2) by the Tautology Theorem]
\itemitem{(4)} $\vdash \x=\x$ \hfill [Identity Axiom]
\itemitem{(5)} $\vdash \f\x_1\dots\x_n = \f\x_1\dots\x_n$ \hfill [from (4) by the Substitution Rule]
\itemitem{(6)} $\vdash \x_i=\y_i \to \f\x_1\dots\x_n = \f\x_1\dots\x_{i-1}\y_i\x_{i+1}\dots\x_n.$ \hfill [from (5) and (3) by the Detachment Rule]

\noindent and expanding for all $i$ we can obtain
$$\eqalign{\vdash \x_1=\y_1 &\to \f\x_1\y_2\dots\y_n = \f\y_1\dots\y_n\cr
\vdash \x_2=\y_2 &\to \f\x_1\x_2\y_3\dots\y_n = \f\x_1\y_2\dots\y_n\cr
&\cdots\cr
\vdash \x_n=\y_n &\to \f\x_1\dots\x_n = \f\x_1\dots\x_{n-1}\y_n.\cr
}$$

Finally, and similar to previous proof (for predicate equality axioms), we use
(i), (ii) and starting with the first formula we find successively
$$\eqalign{\vdash \x_2=\y_2 \to \x_1=\y_1 &\to \f\x_1\x_2\y_3\dots\y_n = \f\y_1\dots\y_n\cr
\vdash \x_3=\y_3 \to \x_2=\y_2 \to \x_1=\y_1 &\to \f\x_1\x_2\x_3\y_4\dots\y_n = \f\y_1\dots\y_n\cr
&\cdots\cr
\vdash \x_n=\y_n \to \cdots \to \x_1=\y_1 &\to \f\x_1\dots\x_n = \f\y_1\dots\y_n\cr
}$$
and the equality axiom for functions follows from the last formula and the Tautology Theorem.
Hence, $T$ and $T'$ have the same theorems since the axioms in which they differ are
provable in the other system, respectively.

\medskip

% ------------------------------------------------------------------------------
\ans 10. 

\ansitem (a)
\itemitem{(1)}  $\vdash \x=\a \to (\A \leftrightarrow \A_\x[\a])$ \hfill [by the Equality Theorem]
\itemitem{(2)}  $\vdash \x=\a \to \A \to \A_\x[\a]$ \hfill [from (1) by the Tautology Theorem]
\itemitem{(3)}  $\vdash \x=\a \to \A_\x[\a] \to \A$ \hfill [from (1) by the Tautology Theorem]
\itemitem{(4)}  $\vdash \A_\x[\a] \to \x=\a \to \A$ \hfill [from (3) by the Tautology Theorem]
\itemitem{(5)}  $\vdash \A_\x[\a] \to \forall\x(\x=\a \to \A)$ \hfill [from (4) by the $\forall$-Introduction Rule]
\itemitem{(6)}  $\vdash \forall\x(\x=\a \to \A) \to (\a=\a \to \A_\x[\a])$ \hfill [by the Substitution Theorem]
\itemitem{(7)}  $\vdash \x=\x$ \hfill [Identity Axiom]
\itemitem{(8)}  $\vdash \a=\a$ \hfill [from (7) by the Substitution Rule]
\itemitem{(9)}  $\vdash \forall\x(\x=\a \to \A) \to \A_\x[\a]$ \hfill [from (6) and (8) by the Tautology Theorem]
\itemitem{(10)} $\vdash \A_\x[\a] \leftrightarrow \forall\x(\x=\a \to \A)$ \hfill [from (5) and (9) by the Tautology Theorem]

\smallskip

\ansitem (b) Consider the first formula when $\A$ is $x=1$ and $\a$ is $x+1$
$$
x+1=1 \leftrightarrow \exists x (x=x+1 \land x=1).
$$
The right-hand side of the implication is always false, but the left-hand side is true
if we pick an instance when $x$ is $0$. Hence this formula is not valid in \N.

Next consider the second formula for the same $\A$ and $\a$
$$
x+1=1 \leftrightarrow \forall x (x=x+1 \to x=1).
$$
In this case, the right-hand side is always true, but the left-hand can be falsified
if we pick an instance when $x$ is $1$ (or any value not equal to 0). Hence this formula
is also not valid in \N.

\smallskip

\medskip

% ------------------------------------------------------------------------------
\ans 11. 

\ansitem (a) Let $\A$ be a formula and let $\B_1, \dots, \B_n$ be the elementary
formulas having an occurrence in $\A$. Let $V_1, \dots, V_m$ all the truth valuations
such that $V_j(\A) = \top$. Let also 
$$
\C^j_i = \cases{\B_i, &if $V_j(\B_i) = \top$\cr
\neg \B_i, &otherwise\cr}
$$
for $1 \le i \le n$ and $1 \le j \le m$. We can now define $\A'$ as
$$
(\C^1_1 \land \dots \land \C^1_n) \lor ... \lor (\C^m_1 \land \dots \land \C^m_n).
$$

To verify that $\A \leftrightarrow \A'$ is a tautology, consider a truth valuation $V$. 
If $V(\A) = \top$, then $V = V_j$ for some $j$, by the definition of $A'$. Moreover,
$V(\C^j_1 \land \dots \land \C^j_n) = V(\C^j_1) \land \dots \land V(\C^j_n) = \top$ and hence $V(\A') = \top$ follows.
If, on the other hand, $V(\A) = \bot$, then $V \ne V_j$ and $V(\C^j_1 \land \dots \land \C^j_n) = \bot$ for all $j$.
\smallskip

\ansitem (b) Let's start from a formula $\B$ in disjunctive form such that $\neg \A \leftrightarrow \B$ is a tautology.
We proceed as in (a): let $\B_1, \dots, \B_n$ be the elementary formulas having an occurrence
in $\A$, let $V_1, \dots, V_m$ all the truth valuations such that $V_j(\neg \A) = \top$ and let
$$
\C^j_i = \cases{\B_i, &if $V_j(\B_i) = \top$\cr
\neg \B_i, &otherwise\cr}
$$
for $1 \le i \le n$ and $1 \le j \le m$. We can now define $\B$ as
$$
(\C^1_1 \land \dots \land \C^1_n) \lor ... \lor (\C^m_1 \land \dots \land \C^m_n)
$$
and we have that $\neg \A \leftrightarrow \B$ is a tautology. Now, by the Tautology Theorem,
we have that if $\neg \A \leftrightarrow \B$ is a tautology, then so are $\neg \neg \A \leftrightarrow \neg \B$
and $\A \leftrightarrow \neg \B$. By expanding $\neg \B$ and applying the fact that $\vdash \neg (\A \lor \B) \leftrightarrow (\neg \A \land \neg \B)$ 
and $\vdash \neg (\A \land \B) \leftrightarrow (\neg \A \lor \neg \B)$ (the De Morgan laws):
$$\eqalign{\neg ((\C^1_1 \land \dots \land \C^1_n) \lor ... \lor (\C^m_1 \land \dots \land \C^m_n))\cr
\neg (\C^1_1 \land \dots \land \C^1_n) \land ... \land \neg (\C^m_1 \land \dots \land \C^m_n)\cr
(\neg \C^1_1 \lor \dots \lor \neg \C^1_n) \land ... \land (\neg \C^m_1 \lor \dots \lor \neg \C^m_n).\cr
}$$

Finally, we can replace each $\neg \C^j_i$ with a new $\D^j_i$ defined as
$$
\D^j_i = \cases{\B_i, &if $\C^j_i = \neg \B_i$\cr
\neg \B_i, &otherwise\cr}
$$
by the Tautology Theorem.

\medskip

% ------------------------------------------------------------------------------
\ans 12. Let's use induction on the length of $\A$ and also consider the defined
symbols $\land$ and $\forall$ as part of the generalized inductive definition of
formula:
\item{$\bullet$} if $\A$ is of the form $\B \lor \C$:
\itemitem{(1)} $\vdash \B^* \leftrightarrow \neg \B$ \hfill [induction hypothesis]
\itemitem{(2)} $\vdash \C^* \leftrightarrow \neg \C$ \hfill [induction hypothesis]
\itemitem{(3)} $\vdash \B^* \land \C^* \leftrightarrow \neg \B \land \neg \C$ \hfill [from (1) and (2) by the Tautology Theorem]
\itemitem{(4)} $\vdash \neg \B \land \neg \C \leftrightarrow \neg (\B \lor \C)$ \hfill [from (3) by the Tautology Theorem (De Morgan laws)]
\itemitem{(5)} $\vdash \B^* \land \C^* \leftrightarrow \neg (\B \lor \C)$ \hfill [from (3) and (4) by the Tautology Theorem]

\item{$\bullet$} if $\A$ is of the form $\B \land \C$ (symmetrical to the previous case):
\itemitem{(1)} $\vdash \B^* \leftrightarrow \neg \B$ \hfill [induction hypothesis]
\itemitem{(2)} $\vdash \C^* \leftrightarrow \neg \C$ \hfill [induction hypothesis]
\itemitem{(3)} $\vdash \B^* \lor \C^* \leftrightarrow \neg \B \lor \neg \C$ \hfill [from (1) and (2) by the Tautology Theorem]
\itemitem{(4)} $\vdash \neg \B \lor \neg \C \leftrightarrow \neg (\B \land \C)$ \hfill [from (3) by the Tautology Theorem (De Morgan laws)]
\itemitem{(5)} $\vdash \B^* \lor \C^* \leftrightarrow \neg (\B \land \C)$ \hfill [from (3) and (4) by the Tautology Theorem]

\item{$\bullet$} if $\A$ is of the form $\exists\x\B$:
\itemitem{(1)} $\vdash \B^* \leftrightarrow \neg \B$ \hfill [induction hypothesis]
\itemitem{(2)} $\vdash \forall\x\B^* \leftrightarrow \forall\x \neg \B$ \hfill [from (1) by the Distribution Rule]
\itemitem{(3)} $\vdash \forall\x\B^* \leftrightarrow \neg\exists\x \neg\neg \B$ \hfill [from (2) expanding the definition of $\forall$]
\itemitem{(4)} $\vdash \forall\x\B^* \leftrightarrow \neg\exists\x \B$ \hfill [from (3) by the Equivalence Theorem and Tautology Theorem]

\item{$\bullet$} if $\A$ is of the form $\forall\x\B$:
\itemitem{(1)} $\vdash \B^* \leftrightarrow \neg \B$ \hfill [induction hypothesis]
\itemitem{(2)} $\vdash \exists\x\B^* \leftrightarrow \exists\x \neg \B$ \hfill [from (1) by the Distribution Rule]
\itemitem{(3)} $\vdash \exists\x\B^* \leftrightarrow \neg\neg \exists\x \neg \B$ \hfill [from (2) by the Tautology Theorem]
\itemitem{(4)} $\vdash \exists\x\B^* \leftrightarrow \neg \forall\x \B$ \hfill [from (3) by the definition of $\forall$]

\item{$\bullet$} if $\A$ is of the form $\neg\B$ and $\B$ is not atomic:
\itemitem{(1)} $\vdash \B^* \leftrightarrow \neg \B$ \hfill [induction hypothesis]
\itemitem{(2)} $\vdash \neg\B^* \leftrightarrow \neg\neg \B$ \hfill [from (1) by the Tautology Theorem]

\item{$\bullet$} if $\A$ is of the form $\neg\B$ and $\B$ is atomic, then 
$\A^*$ is $\A^\circ$ and $\A^\circ$ is $\B$ by definition:
\itemitem{(1)} $\vdash \A^* \leftrightarrow \B$ \hfill [by definition]
\itemitem{(2)} $\vdash \A^* \leftrightarrow \neg\neg\B$ \hfill [from (1) by the Tautology Theorem]
\itemitem{(3)} $\vdash \A^* \leftrightarrow \neg\A$ \hfill [from (2) by the form of $\A$]

\item{$\bullet$} finally, if $\A$ is atomic, then 
$\A^*$ is $\A^\circ$ and $\A^\circ$ is $\neg\A$ by definition:
\itemitem{(1)} $\vdash \A^* \leftrightarrow \neg\A$ \hfill [by definition]

and this concludes the proof.

\vfill
\break
