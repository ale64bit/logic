\leftline{\xmplbx Chapter 3}
\vglue .5\baselineskip 

% ==============================================================================
\beginsection DEFINITIONS

\item{$\bullet$} $\A$ is {\it elementary} if it is either atomic or an instantiation.

\item{$\bullet$} A {\it truth valuation} for $T$ is a mapping from the set of elementary 
formulas in $T$ to the set of truth values.

\item{$\bullet$} $\B$ is a {\it tautological consequence} of $\A_1, \dots , \A_n$ 
if $V(\B) = \top$ for every truth valuation $V$ such that 
$V(\A_1) = \cdots = V(\A_n) = \top$.

\item{$\bullet$} $\A$ is a {\it tautology} if it is a tautological consequence of 
the empty sequence of formulas, i.e. if $V(\A) = \top$ for every truth valuation $V$.

\item{$\bullet$} $\A'$ is an {\it instance} of $\A$ if $\A'$ is of the form $\A_{\x_1, \dots, \x_n}[\a_1, \dots, \a_2]$.

\item{$\bullet$} Let $\A$ be a formula and $\x_1, \dots, \x_n$ its free variables in alphabetical order. The {\it closure}
of $\A$ is the formula $\forall \x_1 \dots \forall \x_n \A$.

\item{$\bullet$} $\A'$ is a {\it variant} of $\A$ if $\A'$ can be obtained from $\A$ by a sequence of replacements of the
following type: replace a part $\exists \x\B$ by $\exists \y\B_\x[\y]$, where $\y$ is a variable not free in $\B$.

\item{$\bullet$} $\A$ is {\it open} if it contains no quantifiers.

\item{$\bullet$} $\A$ is in {\it prenex form} if it has the form $Q\x_1 \dots Q\x_n\B$ 
where each $Q\x_i$ is either $\exists\x_i$ or $\forall\x_i$; $\x_1, \dots, \x_n$ are
distinct; and $\B$ is open.

% ==============================================================================
\beginsection RESULTS

\noindent {\bf \S3.1}

\proclaim Tautology Theorem. If $\B$ is a tautological consequence of $\A_1, \dots, \A_n$,
and $\vdash\A_1, \dots, \vdash \A_n$, then $\vdash \B$.

\proclaim Corollary. Every tautology is a theorem.

\proclaim Lemma 1. If $\vdash \A \lor \B$, then $\vdash \B \lor \A$.

\proclaim Detachment Rule. If $\vdash \A$ and $\vdash \A \to \B$, then $\vdash \B$.

\proclaim Corollary. If $\vdash \A_1, \dots, \vdash \A_n$, and $\vdash \A_1 \to \dots \to \A_n \to \B$, then $\vdash \B$.

\proclaim Lemma 2. If $n \ge 2$, and $\A_1 \lor \cdots \lor \A_n$ is a tautology, then $\vdash \A_1 \lor \cdots \lor \A_n$.

\noindent {\bf \S3.2}

\proclaim $\forall$-Introduction Rule. If $\vdash \A \to \B$ and $\x$ is not free in $\A$, then $\vdash \A \to \forall\x\B$.

\proclaim Generalization Rule. If $\vdash \A$, then $\vdash \forall\x\A$.

\proclaim Substitution Rule. If $\vdash \A$ and $\A'$ is an instance of $\A$, then $\vdash \A'$.

\proclaim Substitution Theorem. 
$$\eqalign{&\vdash \A_{\x_1, \dots, \x_n}[\a_1, \dots, \a_n] \to \exists \x_1 \dots \exists \x_n \A\cr
&\vdash \forall \x_1 \dots \forall \x_n \A \to \A_{\x_1, \dots, \x_n}[\a_1, \dots, \a_n]}$$

\proclaim Distribution Rule. If $\vdash \A \to \B$, then $\vdash \exists\x\A \to \exists\x\B$ and $\vdash \forall\x\A \to \forall\x\B$.

\proclaim Closure Theorem. If $\A'$ is the closure of $\A$, then $\vdash \A'$ iff $\vdash \A$.

\proclaim Corollary. If $\A'$ is the closure of $\A$, then $\A'$ is valid in a structure $\cal A$ iff $\A$ is valid in $\cal A$.

\noindent {\bf \S3.3}

\proclaim Deduction Theorem. Let $\A$ be a closed formula in $T$. For every formula $\B$ of $T$, $\vdash_T \A \to \B$ iff $\B$ is a theorem of $T[\A]$.

\proclaim Corollary. Let $\A_1, \dots, \A_n$ be closed formulas in $T$. For every formula $\B$ in $T$, $\vdash_{T} \A_1 \to \cdots \to \A_n \to \B$ 
iff $\B$ is a theorem of $T[\A_1, \dots, \A_n]$.

\proclaim Theorem on Constants. Let $T'$ be obtained from $T$ by adding new constants (but no new nonlogical axioms).
For every formula $\A$ of $T$ and every sequence $\e_1, \dots, \e_n$ of distinct new constants, 
$\vdash_T \A$ iff $\vdash_{T'} \A[\e_1, \dots, \e_n]$.

\noindent {\bf \S3.4}

\proclaim Equivalence Theorem. Let $\A'$ be obtained from $\A$ by replacing some occurrences of $\B_1, \dots, \B_n$
by $\B_1', \dots, \B_n'$, respectively. If
$$
\vdash \B_1 \leftrightarrow \B_1', \dots, \vdash \B_n \leftrightarrow \B_n'
$$
then
$$
\vdash \A \leftrightarrow \A'.
$$

\proclaim Variant Theorem. If $\A'$ is a variant of $\A$, then $\vdash \A \leftrightarrow \A'$.

\proclaim Symmetry Theorem. $\vdash \a = \b \leftrightarrow \b = \a$.

\proclaim Equality Theorem. Let $\b'$ be obtained from $\b$ by replacing some occurrences of $\a_1, \dots, \a_n$
not immediately following $\exists$ or $\forall$ by $\a_1', \dots, \a_n'$ respectively, and let $\A'$ be
obtained from $\A$ by the same type of replacements. If $\vdash \a_1 = \a_1', \dots \vdash \a_n = \a_n'$ then
$\vdash \b = \b'$ and $\vdash \A \leftrightarrow \A'$.

\proclaim Corollary 1. $\vdash \a_1 = \a_1' \to \cdots \to \a_n = \a_n' \to \b[\a_1, \dots, \a_n] = \b[\a_1', \dots, \a_n']$.

\proclaim Corollary 2. $\vdash \a_1 = \a_1' \to \cdots \to \a_n = \a_n' \to (\A[\a_1, \dots, \a_n] \leftrightarrow \A[\a_1', \dots, \a_n'])$.

\proclaim Corollary 3. If $\x$ does not occur in $\a$, then
$$
\vdash \A_\x[\a] \leftrightarrow \exists\x (\x=\a \land \A)
$$

\vfill
\break

% ==============================================================================
\beginsection EXERCISES

% ------------------------------------------------------------------------------
\ans 1. Let's prove it by induction on theorems (as in \S3.1).
If $\A$ is a theorem provable without use of substitution axioms, 
identity axioms, equality axioms, nonlogical axioms or the $\exists$-introduction 
rule, then it is a tautological consequence of some theorems $\B_1, \dots, \B_n$.
If $n=0$, then $\A$ is a tautology, since it's a tautological consequence of the
empty sequence of formulas. Otherwise, by the induction hypothesis, if 
$\B_1, \dots, \B_n$ can be proven without the use of substitution axioms, identity 
axioms, equality axioms, nonlogical axioms or the $\exists$-introduction rule, 
they are also tautologies. This means that $V(\B_i) = \top$ for all $i$ and truth
valuations $V$, which implies that $V(\A) = \top$ for all truth valuations and thus
$\A$ is also a tautology.
\medskip

% ------------------------------------------------------------------------------
\ans 3. 

\ansitem (a)
Let $\x_1, \dots, \x_n$ the free variables of $\forall\x(\A \to \B)$ and $\exists\x\A \to \exists\x\B$; 
let $T'$ be a theory obtained from $T$ by adding $n$ new constants $\e_1, \dots, \e_n$; 
let $\C$ be $\forall\x(\A \to \B)$ and let $\D$ be $\exists\x\A \to \exists\x\B$.
Note that
$$
\vdash_T \C \to \D \quad\hbox{iff}\quad \vdash_{T'} \C[\e_1, \dots, \e_n] \to \D[\e_1, \dots, \e_n]
$$
by the Theorem on Constants, and
$$
\vdash_{T'} \C[\e_1, \dots, \e_n] \to \D[\e_1, \dots, \e_n] \quad\hbox{iff}\quad \vdash_{T'[\C[\e_1, \dots, \e_n]]} \D[\e_1, \dots, \e_n] 
$$
by the Deduction Theorem. Hence, in $T'[\C[\e_1, \dots, \e_n]]$, we have
\item{} $\vdash \C[\e_1, \dots, \e_n]$ \hfill [the added nonlogical axiom]
\item{} $\vdash \forall\x(\A[\e_1, \dots, \e_n] \to \B[\e_1, \dots, \e_n])$ \hfill [by the definition of $\C$]
\item{} $\vdash \forall\x(\A[\e_1, \dots, \e_n] \to \B[\e_1, \dots, \e_n]) \to (\A[\e_1, \dots, \e_n] \to \B[\e_1, \dots, \e_n])$ \hfill [Substitution Theorem]
\item{} $\vdash \A[\e_1, \dots, \e_n] \to \B[\e_1, \dots, \e_n]$ \hfill [Detachment Rule]
\item{} $\vdash \exists\x\A[\e_1, \dots, \e_n] \to \exists\x\B[\e_1, \dots, \e_n]$ \hfill [Distribution Rule]
\item{} $\vdash \D[\e_1, \dots, \e_n]$ \hfill [by the definition of $\D$]
\smallskip 

\ansitem (b) As in (a), but using the universal-quantifier form of the Distribution Rule.
\medskip

% ------------------------------------------------------------------------------
\ans 5. The existential form
\item{(1)} $\vdash \A \to \exists\x\A$ \hfill [Substitution Theorem or Substitution Axiom]
\item{(2)} $\vdash \A \to \A$ \hfill [Propositional Axiom and definition of $\to$]
\item{(3)} $\vdash \exists\x\A \to \A$ \hfill [$\exists$-Introduction Rule]
\item{(4)} $\vdash \exists\x\A \leftrightarrow \A$ \hfill [from (1) and (3) and the definition of $\leftrightarrow$]

\noindent and the universal form
\item{(1)} $\vdash \forall\x\A \to \A$ \hfill [Substitution Theorem]
\item{(2)} $\vdash \A \to \A$ \hfill [Propositional Axiom and definition of $\to$]
\item{(3)} $\vdash \A \to \forall\x\A$ \hfill [$\forall$-Introduction Rule]
\item{(4)} $\vdash \forall\x\A \leftrightarrow \A$ \hfill [from (1) and (3) and the definition of $\leftrightarrow$]

\medskip

% ------------------------------------------------------------------------------
\ans 6. 

\ansitem (a) 
\item{(1)} $\vdash \A \to \exists\x\exists\y\A$ \hfill [Substitution Theorem]
\item{(2)} $\vdash \exists\x\A \to \exists\x\exists\y\A$ \hfill [$\exists$-Introduction Rule]
\item{(3)} $\vdash \exists\y\exists\x\A \to \exists\x\exists\y\A$ \hfill [$\exists$-Introduction Rule]

The reverse implication is obtained in a similar fashion. Note that it is also possible to obtain 
$\exists\y\exists\x\A$ as a variant of $\exists\x\exists\y\A$: first obtain $\exists\x\exists\x'\A'$
where $\A'=\A_y[\x']$ and $\x'$ is a new variable not appearing in $\A$; then obtain $\exists\y\exists\x'\A'_\x[\y]$ 
and finally $\exists\y\exists\x\A$ by substituting $\x'$ to $\x$. This would imply the result by the
Variant Theorem.

\ansitem (b)
\item{(1)} $\vdash \forall\x\forall\y\A \to \A$ \hfill [Substitution Theorem]
\item{(2)} $\vdash \forall\x\forall\y\A \to \forall\x\A$ \hfill [$\forall$-Introduction Rule]
\item{(3)} $\vdash \forall\x\forall\y\A \to \forall\y\forall\x\A$ \hfill [$\forall$-Introduction Rule]

The reverse implication is obtained in a similar fashion.

\ansitem (c)
\item{(1)} $\vdash \A \to \exists\x\A$ \hfill [Substitution Theorem]
\item{(2)} $\vdash \forall\y\A \to \forall\y\exists\x\A$ \hfill [Distribution Rule]
\item{(3)} $\vdash \exists\x\forall\y\A \to \forall\y\exists\x\A$ \hfill [$\exists$-Introduction Rule]

Note that it might seem that using the dual results in the above proof, the opposite implication could be
obtained (i.e. $\vdash \forall\y\exists\x\A \to \exists\x\forall\y\A$). However this is not the case, as
they result in an alternative proof of the same result as above:

\item{(1)} $\vdash \forall\y\A \to \A$ \hfill [Substitution Theorem]
\item{(2)} $\vdash \exists\x\forall\y\A \to \exists\x\A$ \hfill [Distribution Rule]
\item{(3)} $\vdash \exists\x\forall\y\A \to \forall\y\exists\x\A$ \hfill [$\forall$-Introduction Rule]

\ansitem (d) Consider the formula
$$\forall x \exists y (Sx = y) \to \exists y \forall x (Sx = y).$$
The left side can be interpreted as {\it ``every number has a successor"}, while the
right side can be interpreted as {\it ``there is a number that is the successor of every number"}.


\vfill
\break
