\outline{1} {Chapter 2 First-Order Theories}
\centerline{\xmplbxi CHAPTER 2}
\medskip
\centerline{\xmplbx FIRST-ORDER THEORIES}
\bigskip

% ==============================================================================
\outline{2} {Definitions}
\beginsection DEFINITIONS

\item{$\bullet$} A {\it first-order language} has as symbols:
\itemitem{a)} the {\it variables}: $x$, $y$, $z$, $w$, $x'$, $y'$, $z'$, $w'$, $x''$, $y''$, $z''$, $w''$, \dots
\itemitem{b)} for each $n$, the $n${\it-ary function symbols} and the $n${\it-ary predicate symbols}.
\itemitem{c)} the symbols $\neg$, $\lor$ and $\exists$.
\smallskip

\item{$\bullet$} A {\it term} is defined inductively as:
\itemitem{i)} $\x$ is a term;
\itemitem{ii)} if $\f$ is $n$-ary, then $\f\a_1 \dots \a_n$ is a term.
\smallskip

\item{$\bullet$} A {\it formula} is defined inductively as:
\itemitem{i)} if $\p$ is $n$-ary, then an atomic formula $\p\a_1 \dots \a_n$ is a formula;
\itemitem{ii)} $\neg \A$ is a formula;
\itemitem{iii)} $\lor\A\B$ is a formula;
\itemitem{iv)} $\exists \x\A$ is a formula.
\smallskip

\item{$\bullet$} The {\it height} of a formula is the number of occurrences of $\neg$,
$\lor$ and $\exists$ in the formula.

\item{$\bullet$} A {\it designator} is an expression which is either a term or a formula.
\smallskip

\item{$\bullet$} A {\it structure} $\cal A$ for a first-order language $L$ consist of:
\itemitem{i)} A nonempty set $|{\cal A}|$, the {\it universe} and its {\it individuals}.
\itemitem{ii)} For each $n$-ary function symbol $\f$ of $L$, an $n$-ary function 
$\f_{\cal A} : |{\cal A}|^n \to |{\cal A}|$. (In particular, for each constant 
$\e$ of $L$, $\e_{\cal A}$ is an individual of $\cal A$.)
\itemitem{iii)} For each $n$-ary predicate symbol $\p$ of $L$ other than $=$, an 
$n$-ary predicate $\p_{\cal A}$ in $|{\cal A}|$.

Also, ${\cal A}(\a)$ designates an individual and ${\cal A}(\A)$ designates a truth value.
\smallskip

\item{$\bullet$} A formula $\A$ is {\it valid} in a structure ${\cal A}$ if ${\cal A}(\A') = \top$ for 
every ${\cal A}$-instance $\A'$ of $\A$. In particular, a closed formula $\A$ is valid in
${\cal A}$ iff ${\cal A}(\A) = \top$.
\smallskip

\item{$\bullet$} A formula $\A$ is {\it logically valid} if it's valid in every structure.
\smallskip

\item{$\bullet$} A formula $\A$ is a {\it consequence} of a set $\Gamma$ of formulas if the 
validity of $\A$ follows from the validity of the formulas in $\Gamma$.
\smallskip

\item{$\bullet$} A formula $\A$ is a {\it logical consequence} of a set $\Gamma$ of formulas 
if $\A$ is valid in every structure for $L$ in which all of the formulas in $\Gamma$ are valid.
\smallskip

\item{$\bullet$} A {\it first-order theory} is a formal system $T$ such that
\itemitem{i)} the language of $T$ is a first-order language;
\itemitem{ii)} the axioms of $T$ are the logical axioms of $L(T)$ and certain further axioms, the {\it nonlogical axioms};
\itemitem{iii)} the rules of $T$ are Expansion, Contraction, Associative, Cut and $\exists$-Introduction.
\smallskip

\item{$\bullet$} A {\it model} of a theory $T$, is a structure for $L(T)$ in which all the nonlogical axioms of $T$ are valid.

\item{$\bullet$} A formula $\A$ is {\it valid} in a theory $T$ if it is valid in every model of $T$.

\vfill
\break

% ==============================================================================
\outline{2} {Logical Axioms}
\beginsection LOGICAL AXIOMS

{\leftskip 1in
\itemitem{{\bf Propositional}:} $\neg \A \lor \A$
\itemitem{{\bf Substitution}:} $\A_\x[\a] \to \exists \x\A$
\itemitem{{\bf Identity}:} $\x = \x$
\itemitem{{\bf Equality}:} $\x_1 = \y_1 \to \cdots \to \x_n = \y_n \to \f\x_1 \dots \x_n = \f\y_1 \dots \y_n$
\itemitem{} $\x_1 = \y_1 \to \cdots \to \x_n = \y_n \to \p\x_1 \dots \x_n \to \p\y_1 \dots \y_n$
\par}

% ==============================================================================
\outline{2} {Rules of Inference}
\beginsection RULES OF INFERENCE

{\leftskip 1in
\itemitem{{\bf Expansion}.} Infer $\B \lor \A$ from $\A$.
\itemitem{{\bf Contraction}.} Infer $\A$ from $\A \lor \A$.
\itemitem{{\bf Associative}.} Infer $(\A \lor \B) \lor \C$ from $\A \lor (\B \lor \C)$.
\itemitem{{\bf Cut}.} Infer $\B \lor \C$ from $\A \lor \B$ and $\neg\A \lor \C$.
\itemitem{{\bf $\exists$-Introduction}.} If $\x$ is not free in $\B$, infer $\exists \x\A \to \B$ from $\A \to \B$.
\par}

% ==============================================================================
\outline{2} {Results}
\beginsection RESULTS

\noindent {\bf \S2.4}

\proclaim Lemma 1.
If $\u_1, \dots, \u_n, \u_1', \dots, \u_n'$ are designators
and $\u_1 \dots \u_n$ and $\u_1' \dots \u_n'$ are compatible,
then $\u_i$ is $\u'_i$ for $i = 1, \dots , n$.

\proclaim Formation Theorem.
Every designator can be written in the form $\u\v_1\dots\v_n$, where 
$\bf u$ is a symbol of index $n$ and $\v_1, \dots, \v_n$ are 
designators, in one and only one way.

\proclaim Lemma 2.
Every occurrence of a symbol in a designator $\bf u$ begins an occurrence of a
designator in $\bf u$.

\proclaim Occurrence Theorem.
Let $\bf u$ be a symbol of index $n$, and let $\v_1, \dots, \v_n$ be 
designators. Then any occurrence of a designator $\bf v$ in $\u\v_1 \dots \v_n$
is either all of $\u\v_1 \dots \v_n$ or a part of one of the $\v_i$.

\noindent {\bf \S2.5}

\proclaim Lemma.
Let $\cal A$ be a structure for $L$; $\bf a$ a variable-free term in $L({\cal A})$;
$\bf i$ the name of ${\cal A}(\a)$. If $\bf b$ is a term of $L({\cal A})$ in
which no variable except $\bf x$ occurs, then ${\cal A}({\bf b_x}[\a]) = {\cal A}({\bf b_x}[{\bf i}])$.
If $\A$ is a formula of $L({\cal A})$ in which no variable except $\bf x$ is 
free, then ${\cal A}({\bf A_x}[\a]) = {\cal A}({\bf A_x}[{\bf i}])$.

\anchor{2.5.validity}
\proclaim Validity Theorem.
If $T$ is a theory, then every theorem of $T$ is valid in $T$.

\vfill
\break

% ==============================================================================
\outline{2} {Problems}
\beginsection PROBLEMS

% ------------------------------------------------------------------------------
\ans 1.
\ansitem (a) Let $F(a_1, \dots, a_n)$ be any truth function. We can construct 
another function
$$
F'(a_1, \dots, a_n) = H_{d,m}( H_{c,n}(a_1^1, \dots, a_n^1), \dots, H_{c,n}(a_1^m, \dots, a_n^m) )
$$
where the $a_1^i, \dots, a_n^i$ are all the tuples of truth values such that 
$F(a_1^i, \dots, a_n^i) = \top$. Thus, $a_j^i = a_j$ or $a_j^i = H_\neg(a_j)$, 
for some values of $i$ and $j$. Now, we can see that $F$ and $F'$ are the same 
function, since any truth assignment $a_1', \dots, a_n'$ that satisfies (falsifies) 
$F$, also satisfies (falsifies) $F'$, respectively. This is called 
{\it Disjunctive Normal Form (DNF)}.

We can also construct a similar function
$$
\eqalign{F''(a_1, \dots, a_n) &= H_{c,m}( H_\neg(H_{c,n}(a_1^1, \dots, a_n^1)), \dots, H_\neg(H_{c,n}(a_1^m, \dots, a_n^m)) )\cr
&= H_{c,m}( H_{d,n}(H_\neg(a_1^1), \dots, H_\neg(a_n^1)), \dots, H_{d,n}(H_\neg(a_1^m), \dots, H_\neg(a_n^m)) )}$$
where the $a_1^i, \dots, a_n^i$ are all the tuples of truth values such that 
$F(a_1^i, \dots, a_n^i) = \bot$. It can be seen by a reasoning similar to above, 
that $F$ and $F''$ are the same function. This is called {\it Conjunctive Normal 
Form (CNF)}.
\smallskip
\ansitem (b) It can be seen that
$$\eqalign{
H_{c,n} &= H_\land(a_1, H_\land(a_2, \dots))\cr
H_{d,n} &= H_\lor(a_1, H_\lor(a_2, \dots)).
}$$
This means we can define any truth function $F$ in terms of $H_\neg$, $H_\lor$ 
and $H_\land$, due to (a). Additionally, we can convert each instance of 
$H_\land(a, b)$ into $H_\neg(H_{\lor}(H_\neg(a), H_\neg(b)))$. Thus, every 
truth function is definable in terms of $H_\neg$ and $H_{\lor}$.
\smallskip
\ansitem (c) Since $H_\lor(a, b)$ can be defined as $H_\to(H_\neg(a), b)$, 
every truth function is definable in terms of $H_\neg$ and $H_\to$, due to (b).
\smallskip
\ansitem (d) Since $H_\lor(a, b)$ can be defined as $H_\neg(H_\land(H_\neg(a), H_\neg(b)))$, 
every truth function is definable in terms of $H_\neg$ and $H_\land$, due to (b).
\smallskip
\ansitem (e) Consider the following identities, which can be easily verified 
e.g. via their truth tables
$$
H_\lor(a, a) = a, \quad H_\lor(a, \top) = \top
$$
$$
H_\land(a, a) = a, \quad H_\land(a, \top) = a
$$
$$
H_\to(a, a) = \top, \quad H_\to(a, \top) = \top, \quad H_\to(\top, a) = a
$$
$$
H_\leftrightarrow(a, a) = \top, \quad H_\leftrightarrow(a, \top) = a, \quad H_\leftrightarrow(\top, a) = a.
$$
Thus, any formula consisting of only those connectives and the free variable $a$ 
can be inductively reduced to either $a$ or $\top$ and can never define $H_\neg$. 
Those connectives can only define monotone functions while negation is not 
monotone. Note that allowing constants in the expression would allow to define 
negation as e.g. $H_\neg(a) = H_\to(a, \bot)$.
\medskip

% ------------------------------------------------------------------------------
\ans 2. 
\ansitem (a) Note that $H_d(a, b) = H_\land(H_\neg(a), H_\neg(b))$. We can then define
$$\eqalign{H_\neg(a) &= H_d(a, a)\cr
H_\lor(a, b) &= H_d(H_d(a, b), H_d(a, b))}$$
and thus every truth function is definable in terms of $H_d$ 
(using result from 1.1(b)).
\smallskip
\ansitem (b) Note that $H_s(a, b) = H_\neg(H_\land(a, b))$. We can then define
$$\eqalign{H_\neg(a) &= H_s(a, a)\cr
H_\lor(a, b) &= H_s(H_s(a, a), H_s(b, b))}$$
and thus every truth function is definable in terms of $H_s$ 
(using result from 1.1(b)).
\smallskip
\ansitem (c) Let $H$ be singulary with $H(a_1, \dots, a_n) = H'(a_i)$.
The syntax of every truth function $F(a_1, \dots, a_m)$ definable 
in terms of $H$ can be inductively defined by
$$
e ::= a_j \vert H(e_1, \dots, e_n)
$$
where $1 \le j \le m$ and $e_1, \dots, e_n$ are valid expressions.

We can then reduce every expression to an equivalent expression that 
involves a single $a_j$: as long as the expression has the form $H(e_1, \dots, e_n)$,
we can replace it with $H'(e_i)$ and inductively reduce $e_i$. Thus, every truth 
function $F$ definable in terms of $H$ is singulary and furthermore
$$
F(a_1, \dots, a_m) = {H'}^k(a_j)
$$
for some integers $k \ge 0$ and $1 \le j \le m$.
\smallskip
\ansitem (d) Note that since any $n$-ary truth function is completely 
determined by its truth table, there are $2^{2^n}$ of them. So we know
there are $2^{2^2}=16$ binary truth functions. Let's analyze them:

\itemitem{-} Consider the four binary truth functions $H$ such that
$$
H(a, a) = a.
$$
It is easy to see that any function definable in terms of such $H$ can be 
inductively reduced to $a$, in a similar fashion as before. Thus, none of these four
functions can define every truth function (e.g. negation $H_{\neg}$ cannot
be defined).

\itemitem{-} Consider the four binary truth functions $H$ such that
$$
H(a,a) = \bot.
$$
For each of these four functions, we have
$$
H(a,\bot) \in \{a, \bot\}, \quad H(\bot, a) \in \{a, \bot\}
$$
and thus none of these four functions can define every truth function (e.g. 
negation $H_{\neg}$ cannot be defined).

\itemitem{-} Consider the four binary truth functions $H$ such that
$$
H(a,a) = \top.
$$
This case is symmetric to the previous one. For each of these four functions, we have
$$
H(a,\top) \in \{a, \top\}, \quad H(\top, a) \in \{a, \top\}
$$
and thus none of these four functions can define every truth function (e.g. 
negation $H_{\neg}$ cannot be defined).

\itemitem{-} For the four remaining binary truth functions, we have
$$
H(\top, \top) = \bot, \quad H(\bot, \bot) = \top.
$$
Two of those functions
$$\eqalign{
H_1(\top, \bot) &= \top, \quad H_1(\bot, \top) = \bot\cr
H_2(\top, \bot) &= \bot, \quad H_2(\bot, \top) = \top}
$$
are singulary and thus cannot define functions such as $H_{\lor}$, 
due to the result from 2.2(c). The two remaining functions are 
$H_d$ and $H_s$, presented in 2.2(a) and 2.2(b), respectively.
\medskip

% ------------------------------------------------------------------------------
\ans 3. If $\v$ is empty, then trivially neither $\u$ or $\v'$ are empty, 
and they are both designators.

Let's assume that $\v$ is not empty and that the designator $\u\v$ has the form
${\bf t t}_1 \dots {\bf t}_n$. Since $\u\v$ and $\v\v'$ are 
designators, they both begin with a symbol: thus $\v$ also begins with a symbol, since it
is a non-empty prefix of $\v\v'$. The occurrence of this symbol in $\u\v$ 
begins the occurrence of a designator $\u'$ in $\u\v$ (by Lemma 2), which is 
compatible with $\v$. Moreover, the occurrence of $\u'$ in $\u\v$ is either 
all of $\u\v$ or part of one of the ${\bf t}_i$ (by the Occurrence Theorem). In the 
former case, it means that $\v$ is a designator and $\u$ and $\v'$ are 
empty. On the other hand, if $\u'$ is part of one of the ${\bf t}_i$, it means that
$\v\v'$ begins with $\u'$, and thus $\u'$ and $\v$ are the same
(by the Formation Theorem) and $\v'$ is empty.
\medskip

% ------------------------------------------------------------------------------
\ans 4. If a term is:
\itemitem{i)} a variable $\x'$, then the substitution result is $\x$ itself, which 
is also a term.
\itemitem{ii)} a function application ${\bf fa}_1 \dots \a_n$, then $\a$ is one of 
the $\a_i$ and the substitution result is also a term, or $\a$ is substituted
in one of the terms $\a_i$, and it remains a term, by the induction hypothesis.

If a formula is:
\itemitem{i)} an atomic formula ${\bf pa}_1 \dots \a_n$, then substituting $\a$ in
any of the $\a_i$ results in a term, as previously shown. Thus it remains a formula.
\itemitem{ii)} $\neg \A$, then substituting $\a$ in $\A$ remains a formula by the
induction hypothesis.
\itemitem{iii)} $\lor\A\B$, then substituting $\a$ in $\A$ or $\B$ remains a 
formula by the induction hypothesis.
\itemitem{iv)} $\exists\y\A$, then substituting $\a$ in $\A$ remains a formula 
by the induction hypothesis.
\medskip

% ------------------------------------------------------------------------------
\ans 5. The sections of this problem show that each axiom and rule allows
proving theorems that wouldn't be provable without it. The common strategy
is finding a suitable mapping $f$ from formulas to truth values and a specific 
theorem $\B$, such that if a theorem $\A$ is provable without the axiom or rule in 
question, then $f(\A) = \top$, but $f(\B) = \bot$ and it is thus no provable
without it. This means that none of the axioms or rules are redundant.
\smallskip

\ansitem (a)
The hinted function is defined as:
$$\eqalign{f(\A) &= \top, \quad \hbox{for $\A$ atomic;}\cr
f(\neg \A) &= \bot;\cr
f(\A \lor \B) &= f(\B);\cr
f(\exists \x\A) &= \top.\cr}$$

Let's prove that if $\A$ is provable without propositional axioms then $f(\A) = \top$, by induction on theorems.
In a proof of $\A$, if $\A$ was obtained from:
\itemitem{$\bullet$} a substitution axiom: we have 
$f(\A) = f(\neg \A'_\x[\a] \lor \exists \x\A') = f(\exists \x\A') = \top$;
\itemitem{$\bullet$} an identity axiom: since it's an atomic formula, $f(\A) = f(\x = \x) = \top$;
\itemitem{$\bullet$} an equality axiom: we have $f(\A) =
f(\x_1 = \y_1 \to \x_2 = \y_2 \to \x_1 = \x_2 \to \y_1 = \y_2) =
f(\neg (\x_1 = \y_1) \lor \neg (\x_2 = \y_2) \lor \neg (\x_1 = \x_2) \lor (\y_1 = \y_2)) =
f(\y_1 = \y_2) = \top$;
\itemitem{$\bullet$} the expansion rule: we have $f(\A) = f(\B' \lor \A')$ with $f(\A') = \top$ by the induction hypothesis.
In this case $f(\A) = f(\A') = \top$;
\itemitem{$\bullet$} the contraction rule: we have $f(\A) = f(\A')$ with $f(\A' \lor \A') = \top$ by the induction hypothesis. 
In this case $f(\A' \lor \A') = f(\A') = f(\A) = \top$;
\itemitem{$\bullet$} the associative rule: we have $f(\A) = f((\A' \lor \B') \lor \C')$ with $f(\A' \lor (\B' \lor \C')) = \top$ by the induction hypothesis.
In this case $f(\A' \lor (\B' \lor \C')) = f(\B' \lor \C') = f(\C') = f(\A) = \top$;
\itemitem{$\bullet$} the cut rule: we have $f(\A) = f(\B' \lor \C')$ with $f(\A' \lor \B') = \top$ and $f(\neg \A' \lor \C') = \top$ by the induction hypothesis.
In this case $f(\neg \A' \lor \C') = f(\C') = f(\A) = \top$;
\itemitem{$\bullet$} the $\exists$-introduction rule: we have $f(\A) = f(\exists \x\A' \to \B')$ with $f(\A' \to \B') = \top$ by the induction hypothesis.
In this case $f(\A' \to \B') = f(\neg \A' \lor \B') = f(\B') = f(\A) = \top$.

Thus, if $\A$ is provable without propositional axioms, we have $f(\A) = \top$.
But $f(\neg \neg (x=x) \lor \neg (x=x)) = f(\neg (x=x)) = \bot$ and so it is not provable without propositional axioms.
\smallskip

\ansitem (b)
The hinted function is defined as:
$$\eqalign{f(\A) &= \top, \quad \hbox{for $\A$ atomic;}\cr
f(\neg \A) &= \neg f(\A);\cr
f(\A \lor \B) &= f(\A) \lor f(\B);\cr
f(\exists \x\A) &= \bot.\cr}$$

Let's prove that if $\A$ is provable without substitution axioms then $f(\A) = \top$, by induction on theorems.
In a proof of $\A$, if $\A$ was obtained from:
\itemitem{$\bullet$} a propositional axiom: we have $f(\A) = f(\neg \A' \lor \A') = \neg f(\A') \lor f(\A') = \top$;
\itemitem{$\bullet$} an identity axiom: since it's an atomic formula, $f(\A) = f(\x = \x) = \top$;
\itemitem{$\bullet$} an equality axiom: we have $f(\A) = 
f(\x_1 = \y_1 \to \x_2 = \y_2 \to \x_1 = \x_2 \to \y_1 = \y_2) =
f(\neg (\x_1 = \y_1) \lor \neg (\x_2 = \y_2) \lor \neg (\x_1 = \x_2) \lor (\y_1 = \y_2)) =
\neg f(\x_1 = \y_1) \lor \neg f(\x_2 = \y_2) \lor \neg f(\x_1 = \x_2) \lor f(\y_1 = \y_2) = \top$;
\itemitem{$\bullet$} the expansion rule: we have $f(\A) = f(\B' \lor \A')$ with $f(\A') = \top$ by the induction hypothesis.
In this case $f(\B' \lor \A') = f(\B') \lor f(\A') = f(\A) = \top$;
\itemitem{$\bullet$} the contraction rule: we have $f(\A) = f(\A')$ with $f(\A' \lor \A') = \top$ by the induction hypothesis. 
In this case $f(\A' \lor \A') = f(\A') \lor f(\A') = f(\A') = f(\A) = \top$;
\itemitem{$\bullet$} the associative rule: we have $f(\A) = f((\A' \lor \B') \lor \C')$ with $f(\A' \lor (\B' \lor \C')) = \top$ by the induction hypothesis.
In this case $f(\A' \lor (\B' \lor \C')) = f(\A') \lor f(\B' \lor \C') = f(\A') \lor f(\B') \lor f(\C') =
f(\A' \lor \B') \lor f(\C') = f((\A' \lor \B') \lor \C') = \top$;
\itemitem{$\bullet$} the cut rule: we have $f(\A) = f(\B' \lor \C')$ with $f(\A' \lor \B') = \top$ and $f(\neg \A' \lor \C') = \top$ by the induction hypothesis.
In this case we have $f(\B' \lor \C') = f(\B') \lor f(\C')$, $f(\A' \lor \B') = f(\A') \lor f(\B')$ and
$f(\neg \A' \lor \C') = \neg f(\A') \lor f(\C')$. If $f(\A') = \top$, then $f(\C') = \top$. 
If $f(\A') = \bot$, then $f(\B') = \top$. Thus $f(\B') \lor f(\C') = f(\A) = \top$;
\itemitem{$\bullet$} the $\exists$-introduction rule: we have $f(\A) = f(\exists \x\A' \to \B')$ with $f(\A' \to \B') = \top$ by the induction hypothesis.
In this case $f(\A) = f(\neg \exists \x\A' \lor \B') = \neg f(\exists \x\A') \lor f(\B') = \top$.

Thus, if $\A$ is provable without substitution axioms, we have $f(\A) = \top$. 
But $f(x=x \to \exists x(x=x)) = \neg f(x=x) \lor f(\exists x(x=x)) = \bot$ and so it is not provable without substitution axioms.
\smallskip

\ansitem (c)
The hinted function is defined as:
$$\eqalign{f(\A) &= \bot, \quad \hbox{for $\A$ atomic;}\cr
f(\neg \A) &= \neg f(\A);\cr
f(\A \lor \B) &= f(\A) \lor f(\B);\cr
f(\exists \x\A) &= f(\A).\cr}$$

Let's prove that if $\A$ is provable without identity axioms then $f(\A) = \top$, by induction on theorems.
In a proof of $\A$, if $\A$ was obtained from:
\itemitem{$\bullet$} a propositional axiom: we have $f(\A) = f(\neg \A' \lor \A') = \neg f(\A') \lor f(\A') = \top$;
\itemitem{$\bullet$} a substitution axiom: we have 
$f(\A) = f(\neg \A'_\x[\a] \lor \exists \x\A') = \neg f(\A'_\x[\a]) \lor f(\A') = \top$ (see below for this case);
\itemitem{$\bullet$} an equality axiom: we have $f(\A) = 
f(\x_1 = \y_1 \to \x_2 = \y_2 \to \x_1 = \x_2 \to \y_1 = \y_2) =
f(\neg (\x_1 = \y_1) \lor \neg (\x_2 = \y_2) \lor \neg (\x_1 = \x_2) \lor (\y_1 = \y_2)) =
\neg f(\x_1 = \y_1) \lor \neg f(\x_2 = \y_2) \lor \neg f(\x_1 = \x_2) \lor f(\y_1 = \y_2) = \top$;
\itemitem{$\bullet$} the expansion rule: we have $f(\A) = f(\B' \lor \A')$ with $f(\A') = \top$ by the induction hypothesis.
In this case $f(\B' \lor \A') = f(\B') \lor f(\A') = f(\A) = \top$;
\itemitem{$\bullet$} the contraction rule: we have $f(\A) = f(\A')$ with $f(\A' \lor \A') = \top$ by the induction hypothesis. 
In this case $f(\A' \lor \A') = f(\A') \lor f(\A') = f(\A') = f(\A) = \top$;
\itemitem{$\bullet$} the associative rule: we have $f(\A) = f((\A' \lor \B') \lor \C')$ with $f(\A' \lor (\B' \lor \C')) = \top$ by the induction hypothesis.
In this case $f(\A' \lor (\B' \lor \C')) = f(\A') \lor f(\B' \lor \C') = f(\A') \lor f(\B') \lor f(\C') =
f(\A' \lor \B') \lor f(\C') = f((\A' \lor \B') \lor \C') = \top$;
\itemitem{$\bullet$} the cut rule: we have $f(\A) = f(\B' \lor \C')$ with $f(\A' \lor \B') = \top$ and $f(\neg \A' \lor \C') = \top$ by the induction hypothesis.
In this case we have $f(\B' \lor \C') = f(\B') \lor f(\C')$, $f(\A' \lor \B') = f(\A') \lor f(\B')$ and
$f(\neg \A' \lor \C') = \neg f(\A') \lor f(\C')$. If $f(\A') = \top$, then $f(\C') = \top$. 
If $f(\A') = \bot$, then $f(\B') = \top$. Thus $f(\B') \lor f(\C') = f(\A) = \top$;
\itemitem{$\bullet$} the $\exists$-introduction rule: we have $f(\A) = f(\exists \x\A' \to \B')$ with $f(\A' \to \B') = \top$ by the induction hypothesis.
In this case $f(\A) = f(\neg \exists \x\A' \lor \B') = \neg f(\A') \lor f(\B') = f(\neg \A' \lor \B') = f(\A' \to \B') = \top$.

To treat substitution axioms, let's show that $f(\A_\x[\a]) = f(\A)$ by induction on the length of $\A$:
\itemitem{$\bullet$} for $\A$ atomic with form ${\bf pb}_1 \dots \b_n$: we have $f(\A_\x[\a]) = f({{\bf pb}_1}_\x[\a] \dots {\b_n}_\x[\a]) = \bot$
and $f(\A) = f({\bf pb}_1 \dots \b_n) = \bot$.
\itemitem{$\bullet$} for $\A$ with form $\neg \A'$: we have $f(\A_\x[\a]) = \neg f(\A'_\x[\a])$
and $f(\A) = \neg f(\A')$ and $f(\A'_\x[\a]) = f(\A')$ by the induction hypothesis.
\itemitem{$\bullet$} for $\A$ with form $\A' \lor \B'$: we have $f(\A_\x[\a]) = f(\A'_\x[\a]) \lor f(\B'_\x[\a])$
and $f(\A) = f(\A') \lor f(\B')$ and $f(\A'_\x[\a]) = f(\A')$ and
$f(\B'_\x[\a]) = f(\B')$ by the induction hypothesis.
\itemitem{$\bullet$} for $\A$ with form $\exists \y\A'$: we have $f(\exists \y\A'_\x[\a]) = f(\A'_\x[\a])$
and $f(\exists \y\A') = f(\A')$ and $f(\A'_\x[\a]) = f(\A')$ by the induction hypothesis.

Thus, if $\A$ is provable without identity axioms, we have $f(\A) = \top$. 
But $f(x=x) = \bot$ and so it is not provable without identity axioms.
\smallskip

\ansitem (d) 
The hinted function is defined as:
$$\eqalign{f(\e_i = \e_j) &= \top \quad \hbox{iff $i \le j$;}\cr
f(\neg \A) &= \neg f(\A);\cr
f(\A \lor \B) &= f(\A) \lor f(\B);\cr
f(\exists \x\A) &= \top \quad \hbox{iff $f(\A_\x[\e_i]) = \top$ for some $i$}.\cr}$$

Let's prove that if $\A$ is provable without equality axioms then $f(\A') = \top$ for every formula obtained from $\A$ by 
replacing each variable by some $\e_i$ at all its free occurrences, by induction on theorems.
In a proof of $\A$, if $\A$ was obtained from:
\itemitem{$\bullet$} a propositional axiom: we have $f(\A) = f(\neg \A' \lor \A') = \neg f(\A') \lor f(\A') = \top$ 
for every closed formula $\A''$ obtained from $\A'$ by replacing each variable by some $\e_i$ at all its free occurrences;
\itemitem{$\bullet$} a substitution axiom: we have $f(\A) = f(\neg \A'_\x[\a] \lor \exists \x\A') = \neg f(\A'_\x[\a]) \lor f(\exists \x\A')$.
For every closed formula $\A''$ obtained from $\A'$ by replacing each variable (except $\x$) by some $\e_i$ at all its free occurrences: if
$f(\A''_\x[\e_i]) = \top$ for some $i$, then $f(\exists \x\A'') = \top$ by the definition of $f$. Otherwise, $f(\A''_\x[\e_i]) = \bot$ for all $i$
and thus $\neg f(\A''_\x[\e_i]) = \top$;
\itemitem{$\bullet$} an identity axiom: $f(\A) = f(\x = \x) = \top$ for any substitution of $\x$ by some $\e_i$;
\itemitem{$\bullet$} the expansion rule: we have $f(\A) = f(\B' \lor \A') = f(\B') \lor f(\A')$ with $f(\A') = \top$ 
for every closed formula $\A''$ obtained from $\A'$ by replacing each variable by some $\e_i$ at all its free occurrences, by the induction hypothesis;
\itemitem{$\bullet$} the contraction rule: we have $f(\A) = f(\A')$ with $f(\A' \lor \A') = f(\A') \lor f(\A') = f(\A') = \top$ 
for every closed formula $\A''$ obtained from $\A'$ by replacing each variable by some $\e_i$ at all its free occurrences, by the induction hypothesis;
\itemitem{$\bullet$} the associative rule: we have $f(\A) = f((\A' \lor \B') \lor \C') = f(\A') \lor f(\B') \lor f(\C')$ 
with $f(\A' \lor (\B' \lor \C')) = f(\A') \lor f(\B') \lor f(\C') = \top$ 
for every closed formulas $\A''$, $\B''$ and $\C''$ obtained from $\A'$, $\B'$ and $\C'$, respectively, by replacing each variable by 
some $\e_i$ at all its free occurrences, by the induction hypothesis;
\itemitem{$\bullet$} the cut rule: we have $f(\A) = f(\B' \lor \C') = f(\B') \lor f(\C')$ with $f(\A' \lor \B') = f(\A') \lor f(\B') = \top$ 
and $f(\neg \A' \lor \C') = \neg f(\A') \lor f(\C') = \top$ for every closed formulas $\A''$, $\B''$ and $\C''$ obtained from $\A'$, $\B'$ and $\C'$, 
respectively, by replacing each variable by some $\e_i$ at all its free occurrences, by the induction hypothesis.
If $f(\A') = \top$, then $f(\C') = \top$. If $f(\A') = \bot$, then $f(\B') = \top$. Thus $f(\B') \lor f(\C') = f(\A) = \top$;
\itemitem{$\bullet$} the $\exists$-introduction rule: we have $f(\A) = f(\exists \x\A' \to \B') = \neg f(\exists \x\A') \lor f(\B')$ 
with $f(\A' \to \B') = \neg f(\A') \lor f(\B') = \top$ for every closed formula $\A''$ and $\B''$ obtained from $\A'$ and $\B'$, respectively, 
by replacing each variable by some $\e_i$ at all its free occurrences, by the induction hypothesis. If $f(\B') = \top$, then $f(\A) = \top$ follows 
trivially. Otherwise, we must have $f(\A') = \bot$ for all closed formulas $\A''$ obtained from $\A'$ as described above. This implies that
$f(\exists \x\A') = \bot$ and thus $f(\A) = \top$.

Thus, if $\A$ is provable without equality axioms, we have $f(\A') = \top$ for every formula $\A'$ obtained from $\A$
by replacing each variable by some $\e_i$ at all its free occurences. 
But $f(x=y \to x=z \to x=x \to y=z) = \neg f(x=y) \lor \neg f(x=z) \lor \neg f(x=x) \lor f(y=z) = \bot$ since it does not hold
for the substitution $[\x, \y, \z] \to [\e_1, \e_3, \e_2]$ and so it is not provable without equality axioms.
\smallskip

\ansitem (e)
The hinted function is defined as:
$$\eqalign{f(\A) &= \top, \quad \hbox{for $\A$ atomic;}\cr
f(\neg \A) &= \neg f(\A);\cr
f(\A \lor \B) &= f(\A) \leftrightarrow \neg f(\B);\cr
f(\exists \x\A) &= f(\A).\cr}$$

Let's prove that if $\A$ is provable without the expansion rule then $f(\A) = \top$, by induction on theorems.
In a proof of $\A$, if $\A$ was obtained from:
\itemitem{$\bullet$} a propositional axiom: we have $f(\A) = f(\neg \A' \lor \A') = \neg f(\A') \leftrightarrow \neg f(\A') = \top$;
\itemitem{$\bullet$} a substitution axiom: we have 
$f(\A) = f(\neg \A'_\x[\a] \lor \exists \x\A') = \neg f(\A'_\x[\a]) \leftrightarrow \neg f(\A') = \top$ (see below for this case);
\itemitem{$\bullet$} an identity axiom: since it's an atomic formula, $f(\A) = f(\x = \x) = \top$;
\itemitem{$\bullet$} an equality axiom: we have $f(\A) = 
f(\x_1 = \y_1 \to \x_2 = \y_2 \to \x_1 = \x_2 \to \y_1 = \y_2) =
f(\neg (\x_1 = \y_1) \lor \neg (\x_2 = \y_2) \lor \neg (\x_1 = \x_2) \lor (\y_1 = \y_2)) =
\neg f(\x_1 = \y_1) \leftrightarrow \neg f(\x_2 = \y_2) \leftrightarrow \neg f(\x_1 = \x_2) \leftrightarrow \neg f(\y_1 = \y_2) = \top$;
\itemitem{$\bullet$} the contraction rule: we have $f(\A) = f(\A')$ with $f(\A' \lor \A') = \top$ by the induction hypothesis. 
However, this is a contradiction since $f(\A') \leftrightarrow \neg f(\A') = \bot$ for any $\bf A'$ so it's not possible to have a proof where the contraction rule is applied (???);
\itemitem{$\bullet$} the associative rule: we have $f(\A) = f((\A' \lor \B') \lor \C')$ with $f(\A' \lor (\B' \lor \C')) = \top$ by the induction hypothesis.
In this case $f(\A' \lor (\B' \lor \C')) = f(\A') \leftrightarrow \neg f(\B' \lor \C') = f(\A') \leftrightarrow \neg (f(\B') \leftrightarrow \neg f(\C')) =
f(\A') \leftrightarrow (f(\B') \leftrightarrow f(\C'))$ and $f((\A' \lor \B') \lor \C') = f(\A' \lor \B') \leftrightarrow \neg f(\C') =
(f(\A') \leftrightarrow \neg f(\B')) \leftrightarrow \neg f(\C') = f(\A') \leftrightarrow f(\B') \leftrightarrow f(\C')$;
\itemitem{$\bullet$} the cut rule: we have $f(\A) = f(\B' \lor \C')$ with $f(\A' \lor \B') = \top$ and $f(\neg \A' \lor \C') = \top$ by the induction hypothesis.
In this case $f(\A' \lor \B') = f(\A') \leftrightarrow \neg f(\B')$ and 
$f(\neg \A \lor \C') = \neg f(\A') \leftrightarrow \neg f(\C') = f(\A') \leftrightarrow f(\C')$, and 
thus $f(\B') \leftrightarrow \neg f(\C') = f(\B' \lor \C') = f(\A) = \top$;
\itemitem{$\bullet$} the $\exists$-introduction rule: we have $f(\A) = f(\exists \x\A' \to \B')$ with $f(\A' \to \B') = \top$ by the induction hypothesis.
In this case $f(\neg \A' \lor \B') = f(\A') \leftrightarrow f(\B')$ and $f(\neg \exists \x\A' \lor \B') = f(\A') \leftrightarrow f(\B')$.

To treat substitution axioms, let's show that $f(\A_\x[\a]) = f(\A)$ by induction on the length of $\A$:
\itemitem{$\bullet$} for $\A$ atomic with form ${\bf pb}_1 \dots \b_n$: we have $f(\A_\x[\a]) = f({{\bf pb}_1}_\x[\a] \dots {\b_n}_\x[\a]) = \top$
and $f(\A) = f({\bf pb}_1 \dots \b_n) = \top$.
\itemitem{$\bullet$} for $\A$ with form $\neg \A'$: we have $f(\A_\x[\a]) = \neg f(\A'_\x[\a])$
and $f(\A) = \neg f(\A')$ and $f(\A'_\x[\a]) = f(\A')$ by the induction hypothesis.
\itemitem{$\bullet$} for $\A$ with form $\A' \lor \B'$: we have $f(\A_\x[\a]) = f(\A'_\x[\a]) \leftrightarrow \neg f(\B'_\x[\a])$
and $f(\A) = f(\A') \leftrightarrow \neg f(\B')$ and $f(\A'_\x[\a]) = f(\A')$ and
$f(\B'_\x[\a]) = f(\B')$ by the induction hypothesis.
\itemitem{$\bullet$} for $\A$ with form $\exists \y\A'$: we have $f(\exists \y\A'_\x[\a]) = f(\A'_\x[\a])$
and $f(\exists \y\A') = f(\A')$ and $f(\A'_\x[\a]) = f(\A')$ by the induction hypothesis.

Thus, if $\A$ is provable without the expansion rule, we have $f(\A) = \top$. 
But $f(x=x \lor (\neg (x=x) \lor (x=x))) = f(x=x) \leftrightarrow \neg (\neg f(x=x) \leftrightarrow \neg f(x=x)) = \bot$ and so it is not provable without the expansion rule.
\smallskip

\ansitem (f)
The hinted function is defined as:
$$\eqalign{f(\A) &= \top, \quad \hbox{for $\A$ atomic;}\cr
f(\neg \A) &= \bot;\cr
f(\A \lor \B) &= \top;\cr
f(\exists \x\A) &= \bot.\cr}$$

Let's prove that if $\A$ is provable without the contraction rule then $f(\A) = \top$, by induction on theorems.
In a proof of $\A$, if $\A$ was obtained from:
\itemitem{$\bullet$} a propositional axiom: we have $f(\A) = f(\neg \A' \lor \A') = \top$;
\itemitem{$\bullet$} a substitution axiom: we have 
$f(\A) = f(\neg \A'_\x[\a] \lor \exists \x\A') = \top$;
\itemitem{$\bullet$} an identity axiom: since it's an atomic formula, $f(\A) = f(\x = \x) = \top$;
\itemitem{$\bullet$} an equality axiom: we have $f(\A) = 
f(\x_1 = \y_1 \to \x_2 = \y_2 \to \x_1 = \x_2 \to \y_1 = \y_2) =
f(\neg (\x_1 = \y_1) \lor \neg (\x_2 = \y_2) \lor \neg (\x_1 = \x_2) \lor (\y_1 = \y_2)) = \top$;
\itemitem{$\bullet$} the expansion rule: we have $f(\A) = f(\B' \lor \A')$ with $f(\A') = \top$ by the induction hypothesis.
In this case $f(\A) = f(\B' \lor \A') = \top$;
\itemitem{$\bullet$} the associative rule: we have $f(\A) = f((\A' \lor \B') \lor \C')$ with $f(\A' \lor (\B' \lor \C')) = \top$ by the induction hypothesis.
In this case $f(\A) = f((\A' \lor \B') \lor \C') = \top$;
\itemitem{$\bullet$} the cut rule: we have $f(\A) = f(\B' \lor \C')$ with $f(\A' \lor \B') = \top$ and $f(\neg \A' \lor \C') = \top$ by the induction hypothesis.
In this case $f(\A) = f(\B' \lor \C') = \top$;
\itemitem{$\bullet$} the $\exists$-introduction rule: we have $f(\A) = f(\exists \x\A' \to \B')$ with $f(\A' \to \B') = \top$ by the induction hypothesis.
In this case $f(\A) = f(\neg \exists \x\A' \lor \B') = \top$.

Thus, if $\A$ is provable without the contraction rule, we have $f(\A) = \top$. 
But $f(\neg \neg (x=x)) = \bot$ and so it is not provable without the contraction rule.
\smallskip

\ansitem (g)
The hinted function is defined as:
$$\eqalign{f(\A) &= 0, \quad \hbox{for $\A$ atomic;}\cr
f(\neg \A) &= 1 - f(\A);\cr
f(\A \lor \B) &= f(\A) \cdot f(\B) \cdot (1 - f(\A) - f(\B));\cr
f(\exists \x\A) &= f(\A).\cr}$$

Let's prove that if $\A$ is provable without the associative rule then $f(\A) = 0$, by induction on theorems.
In a proof of $\A$, if $\A$ was obtained from:
\itemitem{$\bullet$} a propositional axiom: we have $f(\A) = f(\neg \A' \lor \A') = 
(1 - f(\A')) \cdot f(\A') \cdot (1 - (1 - f(\A')) - f(\A')) =
(1 - f(\A')) \cdot f(\A') \cdot (f(\A') - f(\A')) = 0$;
\itemitem{$\bullet$} a substitution axiom: we have 
$f(\A) = f(\neg \A'_\x[\a] \lor \exists \x\A') = 
(1 - f(\A'_\x[\a])) \cdot f(\A') \cdot (1 - (1 - f(\A'_\x[\a])) - f(\A')) = 0$ (see below for this case);
\itemitem{$\bullet$} an identity axiom: since it's an atomic formula, $f(\A) = f(\x = \x) = 0$;
\itemitem{$\bullet$} an equality axiom: we have 
$$\eqalign{f(\A) &= f(\x_1 = \y_1 \to \x_2 = \y_2 \to \x_1 = \x_2 \to \y_1 = \y_2)\cr
&= f(\neg (\x_1 = \y_1) \lor \neg (\x_2 = \y_2) \lor \neg (\x_1 = \x_2) \lor (\y_1 = \y_2))\cr
&= (1-f(\x_1=\y_1)) \cdot f(\A') \cdot (f(\x_1 = \y_1) - f(\A'))}$$
$$f(\A') = f(\x_2 = \y_2 \to \x_1 = \x_2 \to \y_1 = \y_2) = (1-f(\x_2 = \y_2)) \cdot f(\A'') \cdot (f(\x_2 = \y_2) - f(\A''))$$
$$f(\A'') = f(\x_1 = \x_2 \to \y_1 = \y_2) = (1-f(\x_1 = \x_2)) \cdot f(\y_1 = \y_2) \cdot (f(\x_1 = \x_2) - f(\y_1 = \y_2)) = 0;$$
\itemitem{$\bullet$} the expansion rule: we have $f(\A) = f(\B' \lor \A')$ with $f(\A') = 0$ by the induction hypothesis.
In this case $f(\A) = f(\B') \cdot f(\A') \cdot (1 - f(\B') - f(\A')) = 0$;
\itemitem{$\bullet$} the contraction rule: we have $f(\A) = f(\A')$ with $f(\A' \lor \A') = 0$ by the induction hypothesis. 
In this case $f(\A' \lor \A') = f(\A') \cdot f(\A') \cdot (1 - f(\A') - f(\A')) = 0$ and the only integer solution is $f(\A') = 0$ and thus $f(\A) = 0$;
\itemitem{$\bullet$} the cut rule: we have $f(\A) = f(\B' \lor \C')$ with $f(\A' \lor \B') = 0$ and $f(\neg \A' \lor \C') = 0$ by the induction hypothesis.
Consider the equations
$$\eqalignno{f(\A' \lor \B') &= f(\A') \cdot f(\B') \cdot (1 - f(\A') - f(\B')) = 0&(1)\cr
f(\neg \A' \lor \C') &= (1 - f(\A')) \cdot f(\C') \cdot (f(\A') - f(\C')) = 0&(2)\cr
f(\A) = f(\B' \lor \C') &= f(\B') \cdot f(\C') \cdot (1 - f(\B') - f(\C')) = 0&(3)
}$$
and the possible cases that satisfy equation (2). First, $(1-f(\A')) = 0$ implies that $f(\A') = 1$ and substituting in equation (1) we obtain $f(\B') \cdot (-f(\B')) = 0$
which means that $f(\B') = 0$ which satisfies equation (3). Second, $f(\C') = 0$, which trivially satifies equation (3). Third, $f(\A') - f(\C') = 0$
which implies $f(\A') = f(\C')$ and substituting in equation (1) we obtain $f(\C') \cdot f(\B') \cdot (1 - f(\C') - f(\B')) = 0$. 
Thus, equation (3) is satisfied in all cases;
\itemitem{$\bullet$} the $\exists$-introduction rule: we have $f(\A) = f(\exists \x\A' \to \B')$ with $f(\A' \to \B') = 0$ by the induction hypothesis.
In this case $f(\A) = f(\neg \exists \x\A' \lor \B') = (1 -f(\A')) \cdot f(\B') \cdot (f(\A') - f(\B')) = f(\A' \to \B') = 0$.

To treat substitution axioms, let's show that $f(\A_\x[\a]) = f(\A)$ by induction on the length of $\A$:
\itemitem{$\bullet$} for $\A$ atomic with form ${\bf pb}_1 \dots \b_n$: we have $f(\A_\x[\a]) = f({{\bf pb}_1}_\x[\a] \dots {\b_n}_\x[\a]) = 0$
and $f(\A) = f({\bf pb}_1 \dots \b_n) = 0$.
\itemitem{$\bullet$} for $\A$ with form $\neg \A'$: we have $f(\A_\x[\a]) = 1 - f(\A'_\x[\a])$
and $f(\A) = 1 - f(\A')$ and $f(\A'_\x[\a]) = f(\A')$ by the induction hypothesis.
\itemitem{$\bullet$} for $\A$ with form $\A' \lor \B'$: we have $f(\A_\x[\a]) = 
f(\A'_\x[\a]) \cdot f(\B'_\x[\a]) \cdot (1 - f(\A'_\x[\a]) - f(\B'_\x[\a]))$
and $f(\A) = f(\A') \cdot f(\B') \cdot (1 - f(\A') - f(\B'))$ and $f(\A'_\x[\a]) = f(\A')$ and
$f(\B'_\x[\a]) = f(\B')$ by the induction hypothesis.
\itemitem{$\bullet$} for $\A$ with form $\exists \y\A'$: we have $f(\exists \y\A'_\x[\a]) = f(\A'_\x[\a])$
and $f(\exists \y\A') = f(\A')$ and $f(\A'_\x[\a]) = f(\A')$ by the induction hypothesis.

Thus, if $\A$ is provable without the associative rule, we have $f(\A) = 0$. 
But $f(\neg (\neg (x=x) \lor \neg (x=x))) = 1 - f(\neg (x=x) \lor \neg (x=x)) = 1 - ((1-f(x=x))^2 \cdot (1 - 2 \cdot (1-f(x=x))))
= 1 - (1 - 2) = 2$ and so it is not provable without the associative rule.
\smallskip

\ansitem (h)
The hinted function is defined as:
$$\eqalign{f(\A) &= \top \quad \hbox{for $\A$ atomic};\cr
f(\neg \A) &= \cases{\top, &if $f(\A) = \bot$ or $\A$ is atomic;\cr
                     \bot, &otherwise.}\cr
f(\A \lor \B) &= f(\A) \lor f(\B);\cr
f(\exists \x\A) &= f(\A).\cr}$$

Let's prove that if $\A$ is provable without the cut rule then $f(\A) = \top$, by induction on theorems.
In a proof of $\A$, if $\A$ was obtained from:
\itemitem{$\bullet$} a propositional axiom: we have $f(\A) = f(\neg \A' \lor \A') = f(\neg \A') \lor f(\A') = \top$
(since if $f(\A') = \bot$, then $f(\neg \A') = \top$ from the definition of $f$);
\itemitem{$\bullet$} a substitution axiom: we have 
$f(\A) = f(\neg \A'_\x[\a] \lor \exists \x\A') = f(\neg \A'_\x[\a]) \lor f(\A') = \top$ (since 
if $f(\A') = \bot$, then $f(\neg \A') = \top$ from the definition of $f$ and $f(\neg \A'_\x[\a]) = f(\neg \A')$. See below for this case);
\itemitem{$\bullet$} an identity axiom: since it's an atomic formula, $f(\A) = f(\x = \x) = \top$;
\itemitem{$\bullet$} an equality axiom: we have $f(\A) = 
f(\x_1 = \y_1 \to \x_2 = \y_2 \to \x_1 = \x_2 \to \y_1 = \y_2) =
f(\neg (\x_1 = \y_1) \lor \neg (\x_2 = \y_2) \lor \neg (\x_1 = \x_2) \lor (\y_1 = \y_2)) =
f(\neg (\x_1 = \y_1)) \lor f(\neg (\x_2 = \y_2)) \lor f(\neg (\x_1 = \x_2)) \lor f(\y_1 = \y_2) = \top$;
\itemitem{$\bullet$} the expansion rule: we have $f(\A) = f(\B' \lor \A')$ with $f(\A') = \top$ by the induction hypothesis.
In this case $f(\A) = f(\B' \lor \A') = f(\B') \lor f(\A') = \top$;
\itemitem{$\bullet$} the contraction rule: we have $f(\A) = f(\A')$ with $f(\A' \lor \A') = \top$ by the induction hypothesis. 
In this case $f(\A' \lor \A') = f(\A') \lor f(\A') = \top$ and thus $f(\A) = f(\A') = \top$;
\itemitem{$\bullet$} the associative rule: we have $f(\A) = f((\A' \lor \B') \lor \C')$ with $f(\A' \lor (\B' \lor \C')) = \top$ by the induction hypothesis.
In this case $f(\A' \lor (\B' \lor \C')) = f(\A') \lor f(\B' \lor \C') = f(\A') \lor f(\B') \lor f(\C') =
f(\A' \lor \B') \lor f(\C') = f((\A' \lor \B') \lor \C') = \top$;
\itemitem{$\bullet$} the $\exists$-introduction rule: we have $f(\A) = f(\exists \x\A' \to \B')$ with $f(\A' \to \B') = \top$ by the induction hypothesis.
In this case we have $f(\A) = f(\neg \exists \x\A' \lor \B') = f(\neg \exists \x\A') \lor f(\B')$ and $f(\neg \A') \lor f(\B') = \top$.
So either $f(\neg \A') = \top$ or $f(\B') = \top$. In the latter case, it follows trivially that $f(\A) = \top$. In the former case,
note that since $f(\exists \x\A) = f(\A)$ and $\exists \x\A$ is not atomic, then $f(\neg \exists \x\A') = f(\neg \A')$.

To treat substitution axioms, let's show that $f(\A_\x[\a]) = f(\A)$ by induction on the length of $\A$:
\itemitem{$\bullet$} for $\A$ atomic with form ${\bf pb}_1 \dots \b_n$: we have $f(\A_\x[\a]) = f({{\bf pb}_1}_\x[\a] \dots {\b_n}_\x[\a]) = \top$
and $f(\A) = f({\bf pb}_1 \dots \b_n) = \top$.
\itemitem{$\bullet$} for $\A$ with form $\neg \A'$ with $\A'$ atomic: we have $f(\A_\x[\a]) = f(\neg \A'_\x[\a]) = \top$
and $f(\A) = f(\neg \A') = \top$.
\itemitem{$\bullet$} for $\A$ with form $\neg \A'$ with $\A'$ not atomic: we have $f(\A_\x[\a]) = f(\neg \A'_\x[\a])$
and $f(\A) = f(\neg \A')$ and $f(\A'_\x[\a]) = f(\A')$ by the induction hypothesis.
\itemitem{$\bullet$} for $\A$ with form $\A' \lor \B'$: we have $f(\A_\x[\a]) = f(\A'_\x[\a]) \lor f(\B'_\x[\a])$
and $f(\A) = f(\A') \lor f(\B')$ and $f(\A'_\x[\a]) = f(\A')$ and
$f(\B'_\x[\a]) = f(\B')$ by the induction hypothesis.
\itemitem{$\bullet$} for $\A$ with form $\exists \y\A'$: we have $f(\exists \y\A'_\x[\a]) = f(\A'_\x[\a])$
and $f(\exists \y\A') = f(\A')$ and $f(\A'_\x[\a]) = f(\A')$ by the induction hypothesis.

Thus, if $\A$ is provable without the cut rule, we have $f(\A) = \top$. 
But $f(\neg \neg (x=x)) = \bot$ since $f(\neg (x=x)) = \top$ and so it is not provable without the cut rule.
\smallskip

\ansitem (i)
The hinted function is defined as:
$$\eqalign{f(\A) &= \top, \quad \hbox{for $\A$ atomic;}\cr
f(\neg \A) &= \neg f(\A);\cr
f(\A \lor \B) &= f(\A) \lor f(\B);\cr
f(\exists \x\A) &= \top.\cr}$$

Let's prove that if $\A$ is provable without the $\exists$-introduction rule then $f(\A) = \top$, by induction on theorems.
In a proof of $\A$, if $\A$ was obtained from:
\itemitem{$\bullet$} a propositional axiom: we have $f(\A) = f(\neg \A' \lor \A') = \neg f(\A') \lor f(\A') = \top$;
\itemitem{$\bullet$} a substitution axiom: we have 
$f(\A) = f(\neg \A'_\x[\a] \lor \exists \x\A') = \neg f(\A'_\x[\a]) \lor f(\exists \x\A') = \top$;
\itemitem{$\bullet$} an identity axiom: since it's an atomic formula, $f(\A) = f(\x = \x) = \top$;
\itemitem{$\bullet$} an equality axiom: we have $f(\A) = 
f(\x_1 = \y_1 \to \x_2 = \y_2 \to \x_1 = \x_2 \to \y_1 = \y_2) =
f(\neg (\x_1 = \y_1) \lor \neg (\x_2 = \y_2) \lor \neg (\x_1 = \x_2) \lor (\y_1 = \y_2)) =
\neg f(\x_1 = \y_1) \lor \neg f(\x_2 = \y_2) \lor \neg f(\x_1 = \x_2) \lor f(\y_1 = \y_2) = \top$;
\itemitem{$\bullet$} the expansion rule: we have $f(\A) = f(\B' \lor \A')$ with $f(\A') = \top$ by the induction hypothesis.
In this case $f(\A) = f(\B' \lor \A') = f(\B') \lor f(\A') = \top$;
\itemitem{$\bullet$} the contraction rule: we have $f(\A) = f(\A')$ with $f(\A' \lor \A') = \top$ by the induction hypothesis. 
In this case $f(\A' \lor \A') = f(\A') \lor f(\A') = f(\A') = f(\A) = \top$;
\itemitem{$\bullet$} the associative rule: we have $f(\A) = f((\A' \lor \B') \lor \C')$ with $f(\A' \lor (\B' \lor \C')) = \top$ by the induction hypothesis.
In this case $f(\A' \lor (\B' \lor \C')) = f(\A') \lor f(\B' \lor \C') = f(\A') \lor f(\B') \lor f(\C') =
f(\A' \lor \B') \lor f(\C') = f((\A' \lor \B') \lor \C') = \top$;
\itemitem{$\bullet$} the cut rule: we have $f(\A) = f(\B' \lor \C')$ with $f(\A' \lor \B') = \top$ and $f(\neg \A' \lor \C') = \top$ by the induction hypothesis.
In this case we have $f(\B' \lor \C') = f(\B') \lor f(\C')$, $f(\A' \lor \B') = f(\A') \lor f(\B')$ and
$f(\neg \A' \lor \C') = \neg f(\A') \lor f(\C')$. If $f(\A') = \top$, then $f(\C') = \top$. 
If $f(\A') = \bot$, then $f(\B') = \top$. Thus $f(\B') \lor f(\C') = f(\A) = \top$;

Thus, if $\A$ is provable without the $\exists$-introduction rule, we have $f(\A) = \top$. 
But $f(\exists y \neg (x=x) \to \neg (x=x)) = \neg f(\exists y \neg (x=x)) \lor \neg f(x=x) = \bot$ and so it is 
not provable without the $\exists$-introduction rule.
\smallskip

\vfill
\break
