\outline{1} {Appendix A Completeness of Propositional Logic}
\centerline{\xmplbxi APPENDIX A}
\medskip
\centerline{\xmplbx COMPLETENESS OF PROPOSITIONAL LOGIC}
\bigskip

The proof of the \jumplink{3.1.tautology}{Tautology Theorem} provides a direct way of proving the completeness 
of propositional logic, i.e. the fragment of the first-order system which does not
contain quantifiers and employs only propositional variables.

Let's define this fragment more precisely. Consider a {\it propositional language} having as symbols the following:
\item{a)} the {\it propositional variables}: $x, y, z, w, x', y', z', w', x'', \dots$;
\item{b)} the symbols $\neg$ and $\lor$.

In this case, the variables range over the set of truth values $\{\top, \bot\}$ and there are no terms. Instead,
the formulas are defined by the generalized inductive definition:
\item{i)} a variable is a formula.
\item{ii)} if $\u$ is a formula, then $\neg\u$ is a formula.
\item{iii)} if $\u$ and $\v$ are formulas, then $\lor\u\v$ is a formula.

There are no nonlogical symbols in this language. Moreover, all the remarks about defined symbols for
first-order languages apply when appropriate. Any formula in this language can be assigned a unique
truth value in the expected way and it is easy to see that every truth function is definable in this language
(see problem 1.1). Finally, the propositional axiom is the only axiom of the system and the rules are the Expansion Rule, Contraction Rule, 
the Associative Rule and the Cut Rule (i.e. all the rules except the $\exists$-Introduction Rule). Let's
call this formal system $P$.

We can see now that a formula $\A$ of $P$ is valid iff every truth valuation of the propositional
variables makes $\A$ true. This is analogous to a first-order tautology, since in $P$ the propositional
variables stand for the elementary formulas. Thus, given a propositional tautology $\A$ in $P$, the proof of the
\jumplink{3.1.tautology}{Tautology Theorem} and its corollary provide a constructive procedure to
produce a proof $\vdash_P \A$. Together with the \jumplink{2.5.validity}{Validity Theorem}, this
implies that this system is complete: that is, a formula is valid iff it is a theorem.

\vfill
\break
