\leftline{\xmplbx Chapter 3}
\vglue .5\baselineskip 

% ==============================================================================
\beginsection DEFINITIONS

\item{$\bullet$} A formula is {\it elementary} if it is either atomic or an instantiation.

\item{$\bullet$} A {\it truth valuation} for $T$ is a mapping from the set of elementary 
formulas in $T$ to the set of truth values.

\item{$\bullet$} $\B$ is a {\it tautological consequence} of $\A_1, \dots , \A_n$ 
if $V(\B) = \top$ for every truth valuation $V$ such that 
$V(\A_1) = \cdots = V(\A_n) = \top$.

\item{$\bullet$} A formula $\A$ is a {\it tautology} it is a tautological consequence of 
the empty sequence of formulas, i.e. if $V(\A) = \top$ for every truth valuation $V$.

% ==============================================================================
\beginsection RESULTS

\noindent {\bf \S3.1}

\proclaim Tautology Theorem. If $\B$ is a tautological consequence of $\A_1, \dots, \A_n$,
and $\vdash\A_1, \dots, \vdash \A_n$, then $\vdash \B$.

\proclaim Corollary. Every tautology is a theorem.

\proclaim Lemma 1. If $\vdash \A \lor \B$, then $\vdash \B \lor \A$.

\proclaim Detachment Rule. If $\vdash \A$ and $\vdash \A \to \B$, then $\vdash \B$.

\proclaim Corollary. If $\vdash \A_1, \dots, \vdash \A_n$, and $\vdash \A_1 \to \dots \to \A_n \to \B$, then $\vdash \B$.

\proclaim Lemma 2. If $n \ge 2$, and $\A_1 \lor \cdots \lor \A_n$ is a tautology, then $\vdash \A_1 \lor \cdots \lor \A_n$.

\noindent {\bf \S3.2}

\proclaim $\forall$-Introduction Rule. If $\vdash \A \to \B$ and $\x$ is not free in $\A$, then $\vdash \A \to \forall\x\B$.

\proclaim Generalization Rule. If $\vdash \A$, then $\vdash \forall\x\A$.

\vfill
\break
