\outline{1} {Appendix B Henkin's Proof Outline}
\centerline{\xmplbxi APPENDIX B}
\medskip
\centerline{\xmplbx HENKINS'S PROOF OUTLINE}
\bigskip

\item{1.} Reduce the characterization of $T[\Gamma]$ to that of $T$ (\jumplink{4.1.reduction}{Reduction Theorem}).
\item{2.} Reduce the validity form of the \jumplink{4.2.completeness_form1}{Completeness Theorem} to the consistency form.
\item{3.} Show that the restriction of a model $\SA' | L(T)$ is a model of $T$; so if we
obtain a model for an extension, we also obtain a model for the base theory.
\item{4.} Given a consistent theory $T$ we need to find a model via syntactical means.
\item{5.} Let the theorems of $T$ say what is truth of the variable-free terms, which
denote individuals.
\item{6.} Introduce the canonical structure $\SA$ (also called the {\it term structure})
using the equivalence classes of variable-free terms as individuals.
\item{7.} Show that the atomic closed formulas are valid in the canonical structure
iff they are theorems.
\item{8.} The two obstacles for showing that the same applies to general closed formulas
are:
\itemitem{(i)} There might be not enough variable-free terms.
\itemitem{(ii)} There might be formulas which are undecidable.
\item{9.} To handle (i), Henkin theories are introduced which guarantee that a constant
exists for every closed instantiation (also called the {\it witness property}).
\item{10.} Show that if $T$ is a complete Henkin theory, then the canonical structure
for $T$, is a model of $T$.
\item{11.} Show that we can construct a Henkin theory $T_c$ from a consistent theory $T$
using the special constants and axioms, and that $T_c$ is a conservative extension of $T$.
\item{12.} Show that every consistent theory has a simple complete extension (\jumplink{4.2.lindenbaum}{Lindenbaum's Theorem}).

\bigskip

\tikzcd[column sep=huge, row sep=huge]
  L   \arrow[r] \arrow[d,"\hbox{extension}" description] & T \hbox{(consistent)} \arrow[d, "\hbox{\jumplink{4.2.lemma3}{conservative extension}}" description] & {\cal A}|L \arrow[l, "\hbox{\jumplink{4.2.lemma1}{model}}"] \\
  L_c \arrow[r] \arrow[dr, to path=|- (\tikztotarget)] & T_c \hbox{(consistent, Henkin)} \arrow[d, "\hbox{\jumplink{4.2.lindenbaum}{simple extension}}" description] & \\
  & U \hbox{(complete, Henkin)} & {\cal A} \arrow[l, "\hbox{\jumplink{4.2.lemma1}{model}}"] \arrow[uu, "\hbox{restriction}" description]
\endtikzcd

\vfill
\break
